\documentclass[11pt, a4paper]{article} % , draft
\usepackage[utf8]{inputenc}

\usepackage{enumitem} % customiçe item dots etc
\usepackage{textgreek} % obv
\usepackage{physics} % for easy derivative notation
\usepackage{amsmath}
\usepackage{amsthm} %theorems
\usepackage{amssymb}
\usepackage{mathtools} % for matrices with blocks inside
\usepackage[scr=boondoxo]{mathalfa}
\usepackage{pst-node}%
\usepackage{mathrsfs}
\DeclareMathAlphabet{\mathpzc}{OT1}{pzc}{m}{it}

\newcommand{\mc}{\multicolumn{1}{c}}
\newcommand{\R}{\mathbb{R}} % command for real R
\newcommand{\Holo}{\mathcal{H}}
\newcommand{\M}{\mathcal{M}}
\newcommand{\C}{\mathbb{C}}
\newcommand{\N}{\mathbb{N}}
\newcommand{\z}{\mathpzc{s}}
\newcommand{\p}{\mathpzc{r}}
\newcommand{\s}{\mathbb{S}}
\newcommand{\W}{\mathbb{W}}
\newcommand{\U}{\mathscr{U}}
\newcommand{\Lg}{\mathscr{L}}
\newcommand{\x}{\mathcal{X}}

\usepackage{csquotes}
\MakeOuterQuote{"}
\setlength{\parskip}{0.3 cm}


%\usepackage{nath} % authomatic parenthesis stuff
%\delimgrowth=1
\usepackage[left=2cm, right=2cm, top=2.1cm, bottom=2.1cm]{geometry} % set custom margins
\usepackage{graphicx} % to insert figures
\usepackage{grffile}
\graphicspath{{Figures/}} % define the figure folder path
\usepackage{subcaption} % for multiple figures at once each with a caption
\usepackage{multirow} %multirow in tables

\usepackage{caption}
\captionsetup[figure]{font=footnotesize} %adjust caption size
\captionsetup[table]{font=footnotesize} %adjust caption size

\usepackage{booktabs} % for pretty tabs in tables
\usepackage{siunitx} % Required for alignment
\captionsetup{labelfont=bf} % bold face captations

\usepackage{hyperref} % makes every reference a hyperlink
\hypersetup{
    colorlinks=true,
    linkcolor=violet,
    filecolor=[rgb]{0.69, 0.19, 0.38},      
    urlcolor=[rgb]{0.0, 0.81, 0.82},
    citecolor=[rgb]{0.69, 0.19, 0.38}
}

\usepackage{epigraph} % for quotations in teh begginig
\setlength\epigraphwidth{8cm}
\setlength\epigraphrule{0pt}
\usepackage{etoolbox}
\makeatletter
\patchcmd{\epigraph}{\@epitext{#1}}{\itshape\@epitext{#1}}{}{}
\renewcommand{\qedsymbol}{o.\textepsilon.\textdelta}

\newtheorem{prop}{Proposition} %so I can use propositions
\newtheorem{cor}{Corollary} %so I can use corollaries
\newtheorem{defi}{Definition} %so I can use corollaries

\makeatother % all this is for the epigraph
\usepackage{imakeidx} % make index
\makeindex[columns=3, title=Alphabetical Index, intoc]

%\title{\vspace{-2.5cm} {\bf Can we make the Exponential scaling in Time\\ be Linear in Time if Parallelized Exponentially? \\ {\em - Part 2 -}} \vspace{-0.4cm}  }
\title{\vspace{-2cm} {\bf Quantum Dynamics:\\  Mixing Wavefunctions and Trajectories}}
\date{\vspace{-11ex}}
\let\clipbox\relax
\usepackage{adjustbox}
\newcolumntype{?}{!{\vrule width 1.5pt}}
\usepackage{abstract}
\setlength{\absleftindent}{0mm}
\setlength{\absrightindent}{0mm}

\usepackage{listings}
\usepackage{xcolor}
\lstset{language=C++,
                basicstyle=\ttfamily,
                keywordstyle=\color{blue}\ttfamily,
                stringstyle=\color{red}\ttfamily,
                commentstyle=\color{green}\ttfamily,
                morecomment=[l][\color{magenta}]{\#}
    backgroundcolor=\color{black!5}, % set backgroundcolor
    basicstyle=\footnotesize,% basic font setting
}

\begin{document}

\maketitle

\tableofcontents
\pagenumbering{gobble}
\clearpage
\pagenumbering{arabic}
\setcounter{page}{1}
\vspace{-0.3 cm}
%\section{The Objective}
%It is well known that the time dependent Schrödinger Equation (TDSE) that predicts the dynamics of a quantum system is a problem that scales exponentially both in space and in time for increasing dimensionality of the problem. This becomes very obvious when interpreting the wave-funtion in terms of an ensemble of tangentially interacting trajectories of the system. That is, quantum mechanical systems (experiments) depend on all their possible realizations in a way that all the possible trajectories of the system interact repulsively among them due to the quantum potential first described by David Bohm. This means that it is equivalent to think on the wavefunction of the system as an ensemble of an infinitely dense set of exactly equivalent systems forming a fluid where each copy of the system cannot cross the trajectory of any other at the same time (they cannot occupy the same point in configuration space-time) and they still have a repelling force pushing the fluid towards the most homogenenous distribution possibel given the manifold described by the potential energy term. 
%
%This clearly shows that it is impossible to evolve a single one of these trajectories without knowing the whole ensemble. This is the so called Quanutm Wholeness. This means that if we increase the dimensionality of the system, it is not enough to increase the computational complexity linearly. A single dimension more implies that in order to know about one single trajectory we now need to know as many trajectories as we needed for the previous dimensionality multiplied by all the possible positions in a new axis. The number of trajectories we would need to simultaneously compute in order to be able to even compute them (and by the way reconstruct the wave-funtion in tyheir vecinity) increases exponentially. However, it is still not clear that there is no method that could allow us evolve self-consistently in parallel at each time step enough trajectories, such that their evolution is linear in time for increasing number of dimensions (even if it scales exponentially in parallel threads that communicate at each time step).
%
%That is, the question is, can we find a method that allows us to compute a single time step that has a fixed cost (perhaps with soem overheads for parallel communiocation) that transfers the expoenntial complexity to the parallelization? That is, it is clear, that if we try to sequentially compute the necessary number of trajectories to advance a central trajectory, we need exponentially more surrounding trajetcories, thus in the single thread's time we would require exponentially more time. Then, even if we are given as many parallel computation threads as we want, we are not able to compute all the trajectories, because they are not independent and they do influence each other. Still, if we allow a cross talk between them every time step, we could achieve an evolution for them that does not increase the complexity in sequential time (unless for the overhead). This cross talk would account for the qwuantum potential propagation. Osea esto es fundamentalemte posible si consiguiese encontrar cual es el pair-wise quantum potential discreto, que al hacer al infinito tiende a la funcion de onda continua. Si fuese asi con una integracion del sistema de edos infinito (pero cada eq simple) en paralelo actualizando los potenciales para cada uno podrias conseguir resolver cualquier problema quantum many body problem si tuvieses suficientes threads paralellos (uno por cada trayectoria evolucionada). HAbria claramente el problema del cross talk, que seria cada vez mas complicada pero bueno, en si seria eso.
%
%Alternativamente, en vez de intentar hacer que todas las trayectorias sean por igual ecuaciones d eNewton, queiza podrias intentar darle un empujon y evolucionar fks de onda condicionadas y una trayectoria por cada conjunto. Ya que cada CWF es 1D y eso es muy facil de resolver. Si fueses capaz de aproximar la full fk de onda con estas slices en cada dimension mejor que usando las trajs en si pues mejor. Ze en si cada CWF es un ensemble de trayectorias, pero de las cuales en principio solo uan (la central) es en cada tiempo la misma. Osea la pregunta es realmente el qtm wholeness necesita trayectorias que estan super lejos? Claro, la cuestion es que no seras capaz de obtener con un solo set de cwf-s en cdad dimension (una trayectoria) evolucionada al mismo timepo el self-impulso dado por las trayectorias que lo rodean. Aka una sola cwf evolucionada en paralelo no funkiona. En todo caso muchas cwf-s evolucionadas tangentemente si, como las trayectorias. Pero esto por supuesto acabaria siendo un ensemble method tipo quantum trajectory method. 
%
%
%Osea la cuestion es que la velocidad e duna trayectoria de Bohm solo depende de la derivad de sus CWF-s en cada timepo! de las direcciones ortonormales (ze claro, el campo de velocidades es la derivada parcial (en las dirs cartesianas de la accion) y el qtm potential solo depende de la derivada parcial en las dirs cartesianas de la "densidad" local!). Entonces, dado un t, dada la fk onda completa, sacas condicioanndo las CWF. Ahora de las CWF tu puedes computar a donde se mueve la traj de Bohm en el sigueinte teimpo. Ahora la pregunta es, puedes si supieses toda la traj evolucionar un tiempo la CWF? Si pudieses ya estaria reuslto el problema many body. Pero la resuesta es que las ecuaciones que rigen las CWF dependen de la full wavefunction al parecer!
%
%Disclaimer, all the present work will be made for 3 dims but is clearly generalizable to N.
\section*{Objectives}\vspace{-0.2cm}
The present document is a review the panorama we face today when talking about quantum dynamics involving trajectories and wavefunctions. It is especially oriented towards lighting possible paths for the development of new algorithms that could help us surpass the difficulties faced by the current standard methods.


\section*{Guideline}\vspace{-0.2cm}

 In the first part of Section 1, a bird eye view will be registered about the paths one could take to approach quantum dynamics involving wavefunctions and/or trajectories. 

In the second part, for each of the approaches mentioned in the previous part, a set of possible methods to face them will be explored in a coarse-grained mode.

In Section 2, we will present all the interesting (exact) equations we find for each of the approaches, trying to do it in a constructive and didactic way. In the way we will explore several possible definitions of trajectories and wavefunctions in the context of each understanding.



\section*{Section 1: On the Employed Vocabulary}
\addcontentsline{toc}{section}{Section 1: On the Employed Vocabulary}

Throughout the document concepts of the Orthodox, Bohmian, Hydrodynamic and Tangent Universe interpretations will be employed together to give names to the mathematical tools we will employ, so let us have a brief brainstorm on them to set things in place.

Given a fluid extending in a real space $\R^N$ and moving in time, the {\bf Eulerian} frame of the system is the stationary frame of reference with respect to the motion of the continuum of moving fluid elements. That is, for each $\vec{x}\in\R^N$ (or the domain where the fluid is defined), we will know the values of the state variables of the system (like the wavefunction or the action/velocity field and the density) at a fixed spatial point for each time. That is, we will know the fields of interest as a function of $(\vec{x},t)$, disregarding a priori that the fluid, hydrodynamicaly, is composed of a continuum (an uncountably infinite number) of tangent {\bf fluid elements} or {\bf fluid particles}. The {\bf Lagrangian} frame of the system then will be knowing about the values of the fields by knowing them as observed by each fluid element along their trajectories. We can label each fluid particle by a vector $\vec{\xi}\in \R^N$ that can denote for instance, the position in $\R^N$ of the particle at the initial time we consider (given the density of the fluid at the first time). Thanks to the fact that the trajectories of the fluid will never cross each other in $\R^N$, because they will have a repulsive interaction, the label will be a precise tag at all times. There will then exist a map $\vec{x}(t,\vec{\xi})\equiv \vec{x}^{\xi}(t)$, that will give us the position at each time of the particle lanbeled by $\vec{\xi}$. Its trajectory. Of course, this must be invertible for the injectivity of the labeling through time, and thus we could also get $\vec{\xi}(t,\vec{x})$, the label of the particle crossing spatial point $\vec{x}$ as a function of time. Then the Lagrangian frame will give us the value of the relevant fields of the fluid as a function of time and the label $(\vec{\xi},t)$.
 

Within the {\bf Orthodox} interpretation, each of these elements is just a mathematically valid tool, but has no interpretative representation. Only the overall density and relative phases of the field (velocity field variations) have physical significance, which are the only things that can be reflected on experimental observations. 

The {\bf Bohmian} interpretation understands these fluid elements to be all the possible {\bf Bohmian trajectories} of the system, from which only happens to exist one, while the rest composes the so called {\bf pilot wave}, that should be understood as a pure fluid without mentioning its composing elements as having any special significance. This pilot wave drives {\bf the} particle (the only one ontologically existing) by a repulsive interaction in configuration space, just like a leaf in a current (in a $\R^N$ current). The trajectory that is said to exist is the one we observe when observing the quantum system, which happens to be a sample statistically obeying the probability density given by the density field of the pilot wave. This is how Bohmian Mechanics ultimately matches orthodox predictions. It is because of that that in a Bohmian perspective, the elements of the density field could also be seen as the "possible experimental outcomes". Each fluid trajectory is a possible experiment, but in a way that possible experiments interfere between them, even if only one of them is truly existing (this is in the author's opinion the point that makes Bohmian still uncomfortable). That is, the rest of possible experiments that do not exist, which are the pilot wave, which is unobservable, influence what reality is. What is the nature of this pilot wave that is attributed a separate contingency of the particle, its {\em arkhé}...no body seems to know it in this interpretation.

Finally, there is the {\bf Tangent Universe} interpretation, which understands that all of these fluid elements exist on a same ontological contingent basis, as a swarm of possible systems that interact repulsively whenever one of them approaches all of its degrees of freedom to another one, but such that they never cross (thus tangential). That is, each "possible experiment", each possible Bohmian trajectory that interacts with the real trajectory in a physical way, is here understood as actually a physical contingent trajetcory that physically "pushes" the actual trajectory we observe. Then why do we only observe one trajectory of the system? One definite position for each particle in the universe? Precisely because we, as observers, are trapped in one of these trajectories. Or have you ever experienced a superposition? Of course not. And as trajectories never cross, we will always be "trapped" in this trajectory, and will only perceive the rest of "Universes" through the tangent force they exert on each degree of freedom of ours. Our lack of knowledge of the position of all the particles in the Universe, makes us be in one of the possible ones with equal probability, which means that our Universe will be a sample of the relative density they follow. Thus allowing the same predictions as Orthodox. This interpretation gives the same material basis to both the density and the velocity field. No need for an unobservable magic pilot wave. Within this interpretation, other tangent trajectories cannot be observed because we happen to percieve a singular one, and as they never cross, we can only feel them through the quantum pressure they exert on our trajectory (due to the local agglomeration of the Universes having the most similar configuration to ours: those push the particles in our Universe through every degree of freedom of our Universe). Just like dark matter or dark energy, we feel a physical influence of them, their information is implicit on our Universe, it is necessary to predict its behavior and the rules of its motion, but we cannot observe the origin directly. We can still measure clearly their contingent effect on every quantum experiment we perform!.

There is finally a discrete version of the last interpretation, suggesting that in fact, it is not necessary that these tangent Universes are infinitely uncountable. If we have a large enough amount of Universes only interacting between them through a repulsive force acting in proportion of their distance in the whole configuration, we can recover in the limit the quantum potential and quantum dynamics. This however, would result in different predictions to the quantum case for a number of discrete Universes smaller than a certain tolerance. It is yet interesting to consider it for potential numerical methods!
\newpage
\section*{Section 2: A Necessary Panoramas}
\addcontentsline{toc}{section}{Section 2: A Necessary Panoramas}
\subsection*{Panorama of the Approaches We Can Take }
\addcontentsline{toc}{subsection}{Panorama of the Approaches We Can Take }

Let us list the main four approaches we can adopt in the context of quantum dynamics involving trajectories and wavefunctions. From I to IV, the approaches will be ordered according to the relevance of trajectories against the relevance of wavefunctions. Let us consider a general quantum system of $N$ degrees of freedom (they could be $N$ 1D bodies, $N/3$ 3D bodies etc.).

\begin{enumerate}
\item[\bf ( I )] {\bf Only or Mainly a Wavefunction: } We could consider a fully wave picture (a continuous fluid moving in $\R^N$). This implies only considering the full N+1 dimensional wavefunction $\psi(\vec{x},t)$ as unknown. This is what we will call the {\bf Fully Eulerian Picture}. If we involve trajectories in the description these will only be computed {\em a posteriori}. This approach can be understood within Orthodox Quantum Mechanics (if only considering the wavefunction) and within Bohmian Mechanics (BM) or Tangent Universe Mechanics (TUM) (if considering also the {\em a posteriori} trajectories).

\item[{\bf ( II )}]{\bf Wavefunctions and Trajectories in Equal footing: } We could consider a scheme where {\bf part} of the quantum system is considered to be described by a certain fluid in $\R^m$ (in the Lagrangian-frame) and {\bf part} of the system is a manifold continuum of particles (in the Eulerian-frame). This will imply considering one or several waves $\{ \psi(\vec{x}_a, \vec{x}_b^\xi(t), t) \}_\xi$ which will describe their degrees of freedom in the {\bf Eulerian frame} and one or several sets of trajectories $\{\vec{x}_b^\chi(t)\}_\xi$ which will describe the motion of {\bf Lagrangian frame} elements of their degrees of freedom. This approach can mainly be interpreted under the prism of BM or TUM.

\item[\bf ( III )]{\bf Mainly Trajectories: } We still view the quantum system as a fluid, but now the values of the field will exclusively be relevant at the positions of {\bf Lagrangian frame} trajectories. The trajectories of elements of the $\R^N$ continuum $\{\vec{x}^\xi(t)\}_\xi$ will be the main actors and the wavefunction will be only implicitly acting. This approach is as akin to the "Continuum of Tangent Universes" Interpretation as we could get. It is also consistent with BM even if there is no explicit pilot wave. BM would understand these elements as possible Bohmian trajectories (or a granulated pilot wave). The wavefunction is somewhat {\em a posteriori}, even if it is not really true, because we know its values in a moving grid.

\item[\bf ( IV ) ]{\bf Only Trajectories: } We will view the quantum system not as a continuum, not as a continuous distribution of $\R^N$ particles of fluid, but instead we will evolve many discrete particles in $\R^N$ that will feel a repulsive force among them acting on the configuration space of the system. Except for this configuration space interaction, the system will behave classically. Here the wavefunction will only be computed {\em a posteriori} if required. This approach can be understood under the prism of the "Discrete Tangent Universe Interpretation".

\end{enumerate}
In reality, for all the interpretations all the approaches would be considered equally valid in a computational basis, but some approaches for some interpretations would be considered as merely mathematical tools for calculations.


\subsection*{ Panorama of the Methods We Can Study }
\addcontentsline{toc}{subsection}{Panorama of the Methods We Can Study}

The order in which the approaches were presented is also the order in which parallelization seems to be most attainable. It is known that evolving a full fluid or wavefucntion of $N$ degrees of freedom  is a problem with exponentially increasing complexity with dimensions. This exponential barrier in time cannot be linearized if we do not apply any approximation (e.g. the Hermitian approximation) or if we do not use external knowledge about the system (e.g. knowing the eigenstates of the Hamiltonian of the system), or both things at once (the Truncated Born-Huang Expansion of the tensor product of conditional wavefunctions for a particle in a channel). However, we can distribute the computational complexity in parallel threads for which we allow cross-talk. If parallel thread communication has negligible overhead, we could in fact make the problem linear in time if parallelized exponentially. This could be the best-case scenario to face big problems with {\em ab initio} methods. 

In the following sections, we will further comment on the equations used for each of the approaches, but for a first look-over, here are some of the main methods used to solve them numerically:
\begin{enumerate}
\item[\bf ( I )] {\bf Only or Mainly a Wavefunction}: There are lots of fixed grid methods, ranging from using naive finite differences to Crank Nicolson or Runge-Kutta Methods. Also, expressing the wavefunction in a certain function basis and then evolving the coefficients could be considered a method type. Then there are the Spectral and Pseudo-Spectral methods based on changing the Schrödinger Equation to other representations, like the momentum representation, involving the Fourier transform, related conceptually with the basis representation methods. 

Except in the case where we know analytically the Hamiltonian eigenstates or some sub-sytem Hamiltonian eigenstates, in general the approach to the full wavefunction allows no escape from the exponential time barrier and are methods hard to be parallelized.

\item [\bf ( III )] {\bf Mainly Trajectories :} This approach basically consists on a dynamical grid of points that move according to the fluid flow. Each fluid element will know the evaluation of the relevant fields like the polar phase and magnitude of the wavefunction along the trajectory it traces. We will have ordinary differential equations ruling their motion, but some functions will need to be computed from the ensemble at each time. Particles encode the field at the points they are and at the same time, the values of the fluid they discover serve as feedback for them to know how to move according to the pilot wave. It is known in general as the family of Quantum Trajectory Methods (QTM), which was boosted by {\em Wyatt et al.} at the beginning of this century. It has essentially two main variations:
\begin{enumerate}
\item Driving the fluid elements or points of the dynamical grid  according to the joint information given by the field elements they drag. The trajectories are driven by the probability density flow lines, so they shape Bohmian trajectories. One of its problems is that Bohmian trajectories avoid nodal regions of the pilot wave, so the grid gets under-sampled or over-sampled for different regions in an uncontrolled manner. The second problem is that the grid gets very unstructured, which can be problematic to feedback the algorithm using the information of the wave that each element drags.

\item Using adaptive grids is one of the main solutions to the fact that Bohmian trajectories avoid regions that could be of interest. It is based on writing the dynamic equations for what fluid elements perceive of the pilot wave if they follow a user-defined path instead of the fluid flow. For instance it can be chosen such that the fluid elements preserve certain monitor functions in each path, so the grid distorts itself to become denser around high fluctuation regions. Many additional methods like adding a viscosity or friction term are very useful here in order to avoid instabilizing the evolution due to spiky fluctuations of the quantum potential.
\end{enumerate}

Both methods have the problem that in order to compute the time evolution of fluid elements, derivatives of the fields they drive are required. This means that the single value of the field they drive is not enough. In fact this is the reason by which it is necessary to simulate several many trajectroies in parallel with cross talk. In order to cope with this problem three approaches can be taken.

\begin{enumerate}
\item Using the values of the field over the trajectories as an unstructured grid, fit a linear sum of analytic functions (by maximum likelihood, least squares, gradient descent etc.). This sum can be analytically derivated and integrated or else numerically. Alternatively a K nearest-neighbor interpolation could also be very useful, which would avoid the need to fit. Just evaluate the points of interest. This is a very interesting method but makes the time evolution more costly than what initially looked like.

\item Generate dynamical equations for the derivatives of the required field quantities. Then evolve the derivatives of the fields along the trajectories too. This increases the number of partial differential equations in play, but allows to evolve {\bf a single trajectory} fully independently of the rest. Conceptually it seems the most interesting idea for a Bohmian. However, it turns out that when trying to get the equations governing the dynamics of those derivatives, infinite chains of equations coupling higher derivatives with lower are obtained. Thus, approximating a certain maximum degree of them will be required. We will review this in the following section more in detail.

\item Knowing the problem, approximate shapes can be obtained as {\em ansatz} for those derivatives of the fields (for the quantum potential etc.).

\end{enumerate}
All of these methods are in general very parallelizable. Each trajectory can be evolved in parallel if we allow cross-talk in each time. It is possibly the only case in which we can achieve fully parallelizing the many body problem.

\item [\bf ( II )] {\bf Wavefunctions and Trajectories in Equal Footing :} We will have that part of the problem to be solved (the Eulerian one) is similar to case (I) and part (the Lagrangian one) similar to case (III). Therefore, we will have the freedom to use one of the methods mentioned in (I) to solve the partial differential equations of the wavefunctions, mixed with the approaches used for (III) in order to account for derivatives in the axes where we only consider discrete trajectories. We will have control over the degree at which we place more or less weight into one or the other problem. Thus, we could arrive at a compromise that has all the main advantages of both methods but perhaps less of their problems.

Following the discussion in the previous section, the trajectories could be chosen to be Bohmian, if they follow the fluid flow, but could also be chosen to be otherwise, in order to achieve an adaptive grid that explores the regions of configuration space we are most interested on.

Following the same ideas, we will be able to solve the derivative problem in several ways:
\begin{enumerate}
\item Evolve many of these wavefunctions with coupled trajectories in order to be able to rebuild the interesting parts of the Eulerian fields necessary to move the trajectories. This could be done by fitting functions or using nearest neighbor approaches. Exponentially less wavefunctions will be required to be computed for increasing dimensionality of the eulerian part of the wavefunctions. However, they will also be each time more complex to compute. On the other hand, expoentially more will be needed for decreasing dimensionality.

\item Generate dynamical equations for those derivatives in the trajectory axes that can be evolved too along the trajectories. This would allow to evolve a single conditional wavefunction "exactly". It turns out that an infinite chain of equations will emerge here too.

\item Knowing the problem, approximate the problematic terms at the theoretical level, ad hoc for the given system. This is what we tried so far.
\end{enumerate}

Clearly, approach II is the generalization of approach I and III, those last being the two extreme cases. Condition it all or condition nothing.

\item [\bf ( IV )] {\bf Only Trajectories:} In this approach, we can choose a large enough number of configuration space trajectories and evolve them using classical mechanics, introducing the necessary repulsive potential between all the trajectories. If the number is large enough, then the theory will be a good enough approximation of continuum quantum mechanics. The point is that there will be no need for the trajectories to "carry" any information about any wave. They are ontologically sufficient to describe quantum phenomena. If we need information of quantum nature, we just need to see the wavefunction as the ensemble limit of the trajectories. From the moving histogram we can fit a density function and the velocities will provide the action field likewise.

This method is perhaps as parallelized as we could get the problem. It would require cross talk to evolve the coupled system of ordinary differential equations though, but could perhaps be efficiently driven.

\end{enumerate}

% Bai kasu generalerako (inspireta perhaps en el QTM) zein trajectory methoderako:
% - Adaptive gridentzako ekuaziñoak sartun leidu ostien kapitulo hori.
% - Sartun ekuaziñoak dynamical deribatuentzako.

\newpage

\section*{Section 3: The Equations for Quantum Dynamics }
\addcontentsline{toc}{section}{Section 3: The Equations for Quantum Dynamics}
\section*{I . Fully Eulerian Equations: Orthodox QM }
\addcontentsline{toc}{section}{I . Fully Eulerian Equations: Orthodox QM }
Given a closed quantum system of $N$ degrees of freedom in a potential field $U(x, t)$, with $x\in\R^N$, described by a complex wavefunction with real support $\psi(x,t)$, the time evolution of the system is governed by the Schrödinger Equation:

\subsection*{(I.a) The Full Schrödinger Equation}
\addcontentsline{toc}{subsection}{(I.a) The Full Schrödinger Equation}
\begin{equation}\label{SE}
i\hbar \pdv{\psi(x,t)}{t} = \qty[-\sum_{j=1}^N\frac{\hbar^2}{2m_j}\pdv[2]{}{x_j} + U(x,t)]\psi(x,t)
\end{equation}
We define the following operator showing in teh right hand side of the equaiton as the Hamiltonian operator:
\begin{equation}
\hat{H}(x, y, t)=\qty[-\sum_{j=1}^N\frac{\hbar^2}{2m_j}\pdv[2]{}{x_j} + U(x,t)]
\end{equation}
Due to the unitary nature of the Schrödinger Equation's time evolution, the norm of the wavefunction is preserved, such that if at a certain known time $\int^\infty_{-\infty} \psi^\dagger(x,t_0)\psi(x,t_0)dx=1$, then the norm is a constant of motion $\int^\infty_{-\infty} \psi^\dagger(x,t)\psi(x,t)dx=1$ $\forall t>t_0$.

By the Born Rule axiom of Orthodox QM, the quantity $\psi^\dagger\psi=|\psi|^2=:\rho(x,t)$ is the probability density that a spatial observation of the degrees of freedom $x$ follow.

\subsection*{(I.b) The Continuity + The Hamilton-Jacobi Equations}
\addcontentsline{toc}{subsection}{(I.b) The Continuity + The Hamilton-Jacobi Equations}
Writing the wavefunction in polar form $\psi(x,t)=R(x,t)exp(iS(x,t)/\hbar)$ with $R(x,t)$ and $S(x,t)$ real fields (note that $|\psi|^2=R^2=\rho(x,t)$), the Schrödinger Equation can be seen to be coupling in a single complex equation, the pair of real partial differential equations:
\begin{equation}\label{CE}
\pdv{}{t} \rho(x,t)=-\sum_{k=1}^N \pdv{}{x_k}\qty( \rho(x,t)\frac{1}{m_k}\pdv{}{x_k} S(x,t) )
\end{equation}
\begin{equation}\label{HJE}
-\pdv{}{t}S(x,t) = \sum_{j=1}^N \frac{\hbar^2}{2m_j} \qty(\pdv{}{x_j} S(x,t) )^2+ V(x,t)+Q(x,t)
\end{equation}
where:
\begin{equation}\label{QP}
Q(x,t)=-\sum_{j=1}^N \frac{\hbar^2}{2m_j}\frac{1}{R(x,t)}\pdv[2]{}{x_j} R(x,t)
\end{equation}
The unknown real fields $R(x,t)$ and $S(x,t)$ can be seen to have a straight-forward interpretation if we realize that $S(x,t)$ can be identified with Hamilton's principal action function of classical mechanics. If so, then we can define the function:
\begin{equation}
v_k(x,t)=\frac{1}{m_k}\pdv{}{x_k}S(x,t)
\end{equation}
to be the velocity field of a fluid with density $R^2(x,t)=\rho(x,t)$. This would make equation \eqref{CE} be the conitnuity equation ruling the motion of the density $\rho$ due to the veolcity field $v_k$ in the Eulerian frame, and the eqution \eqref{HJE} would be identified with the Hamilton-Jacobi equation. As such, we see that apart from the classical potential $U(x,t)$, the fluid also presents a potential energy-like term \eqref{QP} called the quantum potential. It can be understood as a pressure exerted by regions of peaked density on the regions of relaxed density, exactly as if there was a mutually exclusive repulsive interaction between the fluid elements.

These equations allow us then to define the trajectories of the flow lines of the fluid, the trajectories of the fluid elements, given by the solutions of the ordinary differential equation:
\begin{equation}\label{GL}
v_k(x^\xi(t),t)=\dv{}{t}x^\xi(t)
\end{equation}
such that we define the label of each fluid element $\xi$ as the initial position they had. That is:
\begin{equation}
x^\xi(t=t_0)=\xi
\end{equation}
Due to the fact that equation \eqref{GL} is an ordinary differential equation, the existence and uniqueness theorems for the initial value problem will ensure that these fluid element trajectories never cross each other in configuration space $\R^N$ and thus we will be able to evolve an ensemble of them {\em a posteriori}. Each of these trajectories is a possible Bohmian trajectory in BM (weighted by the density $\rho$, which is a field of the pilot wave). In TUM, each of these trajectories is a "Universe", the relative frequency of which (and thus the probability for it to be ours) is weighted by $\rho$. All this will reduce to the Born Rule.

\subsection*{(I.c) Basis Set Expansions}
\addcontentsline{toc}{subsection}{(I.c) Basis Set Expansions}
\subsubsection*{(I.c.1) Hamiltonian and Sub-Hamiltonian Eigenstate Expansion}
\addcontentsline{toc}{subsubsection}{(I.c.1) Hamiltonian and Sub-Hamiltonian Eigenstate Expansion}

If we shorthand the "main" degrees of freedom $x=(x_1,..,x_m)$ and we set the transverse degrees of freedom $y=(x_{m+1},...x_N)$, we can decompose the full Hamiltonian as:
\begin{equation}
\hat{H}(x, y, t)= -\sum_{j=1}^N\frac{\hbar^2}{2m_j}\pdv[2]{}{x_j}+G(x, y, t)=\sum_{j=m+1}^{N}-\frac{\hbar^2}{2m_j}\pdv[2]{}{x_j}+U(x, y, t)+\sum_{j=1}^{m}-\frac{\hbar^2}{2m_j}\pdv[2]{}{x_j}+V(x,t)
\end{equation}
Where we can define the transversal section Hamiltonian:
\begin{equation}
\hat{H}_x(y,t)=\sum_{j=m+1}^{N}-\frac{\hbar^2}{2m_j}\pdv[2]{}{x_j}+U(x, y, t)
\end{equation} 
We then define the set of eigenstates $\{\Phi^j_x(y,t)\}_j$ with eigenvalues $\{\varepsilon_x(t)\}_j$ to be the solution to:
\begin{equation}
\hat{H}_x(y,t)\Phi^j_x(y,t)=\varepsilon^j_x(t)\Phi^j_x(y,t)
\end{equation}
As we know that the hermiticity of the operator $\hat{H}_x(y,t)$ implies its eigenstates form a complete basis of the space $y$ for all times, we could write any wavefunction as a linear combination of them for each $x$:
\begin{equation}
\Psi(x,y,t)=\sum_j \Lambda^j(x,t) \Phi^j_x(y,t)
\end{equation}
with $\Lambda^j(x,t):= \int_{-\infty}^{\infty}\Phi^{j\ \dagger}_x(y,t) \Psi(x,y,t)dy$ the projection coefficients.

If we introduce this shape into the TDSE, we can obtain the differential equations ruling the shape of the coefficients $\Lambda j(x,t)$ by rearranging and multiplying both sides by $\Phi^{k\ \dagger}(y,t)$ and integrating them over all the domain for $y$. Of course we will use here the orthonormality condition $\int_{-\infty}^{\infty}\Phi^{k\ \dagger}(y,t) \Phi^{j}(y,t) dy= \delta_{kj}$. This leaves the equivalent to the Schrödinger Equation:
\begin{equation}\label{XabEq}
i\hbar \pdv{}{t}\Lambda^k(x,t) = \qty( \varepsilon^k(x,t) + \sum_{s=1}^m\frac{\hbar^2}{2m_s}\pdv[2]{}{x_s}+V(x,t))\Lambda^k(x,t)+
\end{equation}
$$
 +\sum_j \qty{ W^{kj}(x,t) + \sum_{s=1}^m S^{kj}_s(x,t)+F^{kj}_s(x,t)\pdv{}{x_s} } \Lambda^j(x,t) 
$$
where we have defined the coupling terms between the transversal section eigenstates:
\begin{equation}
W^{kj}(x,t) = -i\hbar\int_{-\infty}^{\infty}\Phi_x^{k\dagger}(y,t) \pdv{\Phi^j_x(y,t)}{t} dy
\end{equation}
\begin{equation}
S^{kj}_s(x,t) = -\frac{\hbar^2}{2m_s}\int_{-\infty}^{\infty}\Phi_x^{k\dagger}(y,t) \pdv[2]{}{x_s} [\Phi^j_x(y,t)] dy
\end{equation}
\begin{equation}
F^{kj}_s(x,t) = -\frac{\hbar^2}{m_s}\int_{-\infty}^{\infty}\Phi_x^{k\dagger}(y,t) \pdv{}{x_s}\Phi^j_x(y,t) dy
\end{equation}

Which is a coupled linear partial differential equation for the $m$ dimensional $\Lambda^j(x,t)$ coefficients, that requires the knowledge of the $N-m$ dimensional transversal section eigenstates $\Phi^k_x(y,t)$ and their coupling integrals $W^{kj}, S^{kj}_s, F^{kj}_s$. These coupling terms can be simplified if the transversal Hamiltonian varies very gently in $x$ and/or $t$ (so called adiabatically).

If the eigenstates are known analytically, then the problem has a complexity only due to the $m$ spatial dimension coefficients $\Lambda^j(x,t)$, which can range from $m=0$ (and $N-m=N$) to $m=N$ (and $N-m=0$), respectively: only coefficients that vary in time (and eigenstates of the full Hamiltonian) and the full Schrödinger Equation (with no eigenstate).

\subsubsection*{(I.c.2) Arbitrary known orthonormal Base Expansion}
\addcontentsline{toc}{subsubsection}{(I.c.2) Arbitrary known orthonormal Base Expansion}

We will build here the analogue of the generalized method \eqref{XabEq} of the previous section. Using the notation $x=(x_1,..,x_m)$ and $y=(x_{m+1},...,x_N)$, we will assume we know an arbitrary orthonormal set of functions $\{ f^j(y,t) \}_j$ spanning the space $y$. They need not depend on time, but for generality we will consider so (the difference will be that the coupling terms would simplify). For the completeness of the basis for the subspace, we could find coefficients $\Lambda^j(x,t)$ such that:
\begin{equation}
\psi(x,y,t)=\sum_j \Lambda^j(x,t) f^j(y,t)
\end{equation}
Note that unlike the transversal section eigenstates, these do not depend on $x$!

Then introducing this ansatz into the full TDSE, we can get the dynamic equations for the coefficients $\Lambda^j(x,t)$. By using the orthonormality condition $\int_{-\infty}^{\infty}f^{k\dagger}(y,t)f^k(y,t)dy=1$ we can get:
\begin{equation}
i\hbar \pdv{}{t}\Lambda^j(x,t)= \sum_{s=1}^m \hat{T}_m \Lambda^j(x,t) + \sum_k \qty( W^{jk}(t)+\sum_{r=m}^NS^{jk}_r(t)+D^{jk}(x,t) )\Lambda^k(x,t)
\end{equation}
with:
\begin{equation}
W^{jk}(t):=\int_{-\infty}^{\infty}f^{j\dagger}(y,t) \pdv{f^k(y,t)}{t} dy
\end{equation}
\begin{equation}
S^{jk}_r(t):=\int_{-\infty}^{\infty}f^{j\dagger}(y,t) \hat{T}_{x_s}[f^k(y,t)] dy
\end{equation}
\begin{equation}
D^{jk}(x,t):=\int_{-\infty}^{\infty}f^{j\dagger}(y,t)U(x,y,t) f^k(y,t) dy
\end{equation}
Note that if the orthonormal vectors were chosen to be time independent then $W^{jk}(t)$ would vanish and $S^{jk}(t)$ would be time independent. 

We achieve a similar equation to \eqref{XabEq}, where the only task we would need would be to compute the coupling integrals and then evolve the coupled linear system of equations for the coefficients $\Lambda^j(x,t)$. The integrals would be $N-m$ dimensional, while the coupled system would evolve fields with $m$ spatial dimensions. 

If we knew analytically the orthonormal functions, we could be able to compute the integrals symbolically, which would allow us to safe the numerical integration and the complexity would be left to the equation system's, which is the $m$ we fix from $\{0,1,2,...,N\}$.

In this case though, we will have lost the possibility to study the adiabaticity and to approximate the coupling terms in consequence. Perhaps, the number of required $J$ will also increase relative to the case in which we used eigenstates of transversal sections.



\subsection*{(I.d) Dynamic Equations for Partial Differentials}
\addcontentsline{toc}{subsection}{(I.d) Dynamic Equations for Partial Differentials}
In the next section, we will find the advantage of having dynamic equations not only for the main waves $\psi$ or $S$ and $R$, but also for their derivatives in space.

\subsubsection*{(I.d.1) For the Wavefunction}
\addcontentsline{toc}{subsubsection}{(I.d.1) For the Wavefunction}
If we take the Schrödinger Equation \eqref{SE} and partial derviate it in $x_k$ at each side and we assume the wavefunction is regular enough in all its variables $t$, $\vec{x}$ in order to use Schwartz's Law for crossed partial derivatives, we get:
\begin{equation}
i\hbar \pdv{}{t}\qty[\pdv{}{x_k}\psi(x,t)] = \sum_{j=1}^N\frac{-\hbar^2}{2m_j}\pdv[2]{}{x_j}\qty[\pdv{}{x_k}\psi(x,t)]+U(x,t)\pdv{}{x_k}\psi(x,t) + \psi(x,t)\pdv{}{x_k}U(x,t)
\end{equation}
Meaning that the function $\psi^{(1)}_k(x,t):=\pdv{}{x_k}\psi(x,t)$ evolves in time just as a Schrödinger Equation, but with an added non-linearity involving its primitive in $x_k$. That is, we could actually evolve the dynamics of the first partial derivatives if we coupled them with the evolution of the wavefunction.

If we repeat the trick, we can get a dynamical equation for the second partial derivative of the wavefunction in space.
\begin{equation}
i\hbar \pdv{}{t}\qty[\pdv[2]{}{x_k}\psi(x,t)] = \sum_{j=1}^N\frac{-\hbar^2}{2m_j}\pdv[2]{}{x_j}\qty[\pdv[2]{}{x_k}\psi(x,t)]+U(x,t)\pdv[2]{\psi(x,t)}{x_k} + \psi(x,t)\pdv[2]{}{x_k}U(x,t)+2\pdv{\psi(x,t)}{x_k}\pdv{}{x_k}U(x,t)
\end{equation}
Defining $\psi^{(j)}_k(x,t):=\pdv[j]{}{x_k}\psi(x,t)$ this means:
\begin{equation}
i\hbar \pdv{}{t}\psi^{(2)}_k(x,t) = \qty(\sum_{j=1}^N\frac{-\hbar^2}{2m_j}\pdv[2]{}{x_j}+U(x,t))\psi^{(2)}_k(x,t)+ \psi(x,t)\pdv[2]{}{x_k}U(x,t)+2\psi^{(1)}_k(x,t)\pdv{}{x_k}U(x,t)
\end{equation}
Which could be easily generalized to get the dynamical equation for any superior spatial derivative. It is evident that the dynamic equation for $\psi^{(j)}_k(x,t)$ would also involve in the same equation all $\psi^{(s)}_j(x,t)$ with $s<k$. However, it will also include always a second spatial derivative of $\psi^{(s)}_j(x,t)$. Meaning the term $\psi^{(s+2)}_j(x,t)$ will always be coupled to $\psi^{(s)}_j(x,t)$ and at the same time $\psi^{(s+4)}_j(x,t)$ will be coupled to the first. Thus, an inifnite number of partial differential equations in time should be solved in order to avoid explicitly derivating any function in space. 

What we could do though, in order to avoid having to solve an endless sequence of partial derivatives, is to assume that at some point $\pdv[J]{}{x_a}\psi(x,t)\simeq 0$, which seems reasonable for a big enough $J$. If we assumed so, then we would be left with a coupled linear system of equations ruling the dynamics of the functions $\psi^{(1)}_k(x,t),\psi^{(1)}_k(x,t)...\psi^{(J)}_k(x,t)$ and all of their second partial derivatives for the kinetic part (which could themselves be obtained with the same method by augmenting the number of equations).

Clearly, in the Eulerian frame using these equations makes almost no sense, but their use will become evident when we go into the Lagrangian frame.

\subsubsection*{(I.d.2) For the Density and Action}
\addcontentsline{toc}{subsubsection}{(I.d.2) For the Density and Action}

Doing the same for the Hamilton-Jacobi and the Continuity Equations \eqref{HJE} and \eqref{CE}, will turn out to be more dramatic. Due to their non-linear nature, each time higher derivative terms will emerge. Again we will have an infinite chain of partial differential equations.

To see, this, let us take the derivative in $x_k$ at side and side in both equations and rearrange the terms to get:
\begin{equation}
\pdv{}{t} \qty[\pdv{}{x_k}\rho(x,t)]=-\sum_{k=1}^N \pdv{}{x_k}\qty( \qty[ \pdv{}{x_k}\rho(x,t)]\frac{1}{m_k}\pdv{}{x_k} S(x,t)+\rho(x,t)\frac{1}{m_k}\pdv{}{x_k} \qty[\pdv{}{x_k}S(x,t)] )
\end{equation}
\begin{equation}
-\pdv{}{t}\qty[\pdv{S(x,t)}{x_k}] = \sum_{j=1}^N \frac{\hbar^2}{m_j}\pdv{S(x,t)}{x_j} \pdv{}{x_j}\qty[\pdv{}{x_k} S(x,t) ]+ \pdv{}{x_k}V(x,t)+\pdv{}{x_k}Q(x,t)
\end{equation}
with:
\begin{equation}
\pdv{}{x_k}Q(x,t)=-\sum_{j=1}^N \frac{\hbar^2}{2m_j}\qty( -\frac{1}{R^2(x,t)}\qty[\pdv{R(x,t)}{x_k}]\pdv[2]{}{x_j} R(x,t)+\frac{1}{R(x,t)}\pdv[2]{}{x_j} \qty[\pdv{}{x_k}R(x,t)])
\end{equation}

If we repeat the procedure for higher derivatives, we will obtain again a set of coupled partial differential equations (non-linear in this case), which can only be made a finite number of equations if we assume that for a certain $J$ $\pdv[J]{}{x_k}R(x,t)\simeq 0$ and $\pdv[J]{}{x_k}S(x,t)\simeq 0$ $\forall k$ and some crossed partial derivatives.


\section*{III . Fully Lagrangian Equations: Bohmian QM}
\addcontentsline{toc}{section}{III . Fully Lagrangian Equations: Bohmian QM}
Given we parametrize the fluid elements with labels $\vec{\xi}\in\R^N$ referring to their initial position $\vec{\xi}=x(t_0, \vec{\xi})$, we define the set of trajectories of the continuum as $\vec{x}(t;\vec{\xi}) \equiv \vec{x}^\xi(t)$. We will then denote by $\vec{x}_b=(x_1,...,x_{a-1}, x_{a+1},...,x_N)$ the set of degrees of freedom excluding the $a-th$ $x_a$.
\subsection*{(III.a) The Schrödinger Equation}
\addcontentsline{toc}{subsection}{(III.a) The Schrödinger Equation}
If we evaluate the position in the Lagrangian frame $\vec{x}(t;\vec{\xi})$ in the Schrödinger Equation \eqref{SE}:
\begin{equation}
i\hbar \pdv{}{t}\psi(\vec{x}^\xi(t),t)=-\sum_{a=1}^N \frac{\hbar^2}{2m_a}\pdv[2]{}{x_a}\psi(x_a, \vec{x}_b^\xi(t),t)\Big\rvert_{x^\xi_a(t)}+U(\vec{x}_b^\xi(t),t)\psi(\vec{x}^\xi(t),t)
\end{equation}
Using that by the chain rule:
\begin{equation}
\dv{}{t}\psi(\vec{x}^\xi(t), t)=\pdv{}{t}\psi(\vec{x}^\xi(t), t)+\sum_a \pdv{}{x_A}\psi(x_a,\vec{x}_b^\xi(t), t)\Big\rvert_{x^\xi_a(t)}\cdot \dv{}{t}x^\xi_a(t)
\end{equation}
We arrive at:
\begin{equation}\label{KinAdv}
i\hbar \dv{}{t}\psi(\vec{x}^\xi(t),t)=-\sum_{a=1}^N \frac{\hbar^2}{2m_a}\pdv[2]{}{x_a}\psi(x_a, \vec{x}_b^\xi(t),t)\Big\rvert_{x^\xi_a(t)}+U(\vec{x}^\xi(t),t)\psi(\vec{x}^\xi(t),t)+i\hbar \sum_{a=1}^N  \pdv{}{x_a}\psi(x_a,\vec{x}_b^\xi(t), t)\Big\rvert_{x^\xi_a(t)}\cdot \dv{}{t}x^\xi_a(t)
\end{equation}
If we now impose that the trajectories follow the velocity field given by the derivative of the phase of the wavefunction:
\begin{equation}\label{ImpBohm}
\dv{}{t}x_a^\xi(t)=\frac{1}{m_a}\pdv{}{x_a}S(x_a, \vec{x}_b^\xi(t),t)\Big\rvert_{x^\xi_a(t)}=\frac{\hbar^2}{m_a}\mathbb{I}m \qty( \psi^{-1}(\vec{x}^\xi(t),t)\pdv{}{x_a}\psi(x_a, \vec{x}_b^\xi(t),t)\Big\rvert_{x^\xi_a(t)})
\end{equation}
 we will have that the dynamical equation will provide us the time evolution of Bohmian trajectories. Note that we could have chosen an alternative velocity field too.
 
Then we have that equation \eqref{KinAdv} describes the motion in time of the wavefunction in the Lagrangian frame, that is, the value of the wavefunction that each fluid element $\vec{\xi}$ observes in time, because note that $\psi(\vec{x}^\xi(t),t)=\psi(t, \vec{\xi})$.

If we now define the terms, Kinetic and Advective Correlation Potentials as:
\begin{equation}\label{Kin}
K(\vec{x},t)=\sum_{a=1}^N \frac{\hbar^2}{2m_a}\pdv[2]{}{x_a}\psi(\vec{x},t)
\end{equation}
\begin{equation}\label{Adv}
A(\vec{x}^\xi(t),t)=i\hbar \sum_{a=1}^N \pdv{}{x_a}\psi(x_a,\vec{x}_b^\xi(t), t)\Big\rvert_{x^\xi_a(t)}\cdot \dv{}{t}x^\xi_a(t)
\end{equation}
Equation \eqref{KinAdv} could be alternatively written as:
\begin{equation}
i\hbar \dv{}{t}\psi(\vec{x}^\xi(t),t)=U(\vec{x}^\xi(t),t)\psi(\vec{x}^\xi(t),t)+A(\vec{x}^\xi(t),t)+K(\vec{x}^\xi(t),t)
\end{equation}

Evolving the value of $\psi(t,\vec{\xi})$ for a grid of fluid elements $\{\vec{\xi}_k\}_{k=1}^M$ given equation \eqref{KinAdv} has the clear problem that we require the knowledge of the partial derivatives in space for the wavefunction at each time. However, we only know the value of the wavefunction at the points $\{\vec{x}(\vec{\xi}_k,t)\}_{k=1}^M$, which might be a structured cartesian grid at $t=t_0$ if we choose $\{\vec{\xi}_k\}_{k=1}^M$ to be so, but will become into an unstructured grid very fast, because each fluid element will follow the fluid flow given by equation \eqref{ImpBohm}. Thus, on the one hand, we will not be able to evolve a single fluid element, because we need first order local information in the surrounding of the fluid element in order to get the derivative in each of the directions $x_a$, and we only have information at a "zeroth order" (at the same point). In order to compute this derivatives, we will therefore require to compute them numerically on an unstructured mesh formed by the fluid elements. For this, we could fit an analytical function sum to the Lagrangian frame wavefunction as we explained in the second section. The point is that, many fluid elements will be required to be evolved simultaneously in a way that they affect each other's next value of the wavefunction and next point in their trajectory.

A very interesting alternative that would allow us to evolve a single fluid element would be to also have a dynamical equation for the values of the spatial derivatives along the trajectories of the elements! This is exactly what we seek in the next sub-section.

It would also be interesting to write down equation \eqref{KinAdv} in terms of partial derivatives in the label space (initial positions) $\vec{\xi}$, which we can make sure it will be a regular Cartesian mesh. However, if try this, we will realize that we require the Jacobian of the density, that is:
\begin{equation}
\pdv{f(\vec{\xi},t)}{x_k}=\sum_j \pdv{f(\vec{\xi},t)}{\xi_j}\pdv{\xi_j(x,t)}{x_j}
\end{equation}
\subsubsection*{(III.a.1.1) Dynamics of Partial Derivatives}
\addcontentsline{toc}{subsubsection}{(III.a.1.1) Dynamics of Partial Derivatives For the Wavefunction}

\subsubsection*{(III.a.1.2) Partial Derivatives relative to the Labels}
\addcontentsline{toc}{subsubsection}{(III.a.1.2)Partial Derivatives relative to the Labels}

\subsubsection*{(III.a.1.3) Adaptive Grid Equations}
\addcontentsline{toc}{subsubsection}{(III.a.1.3) Adaptive Grid Equations}
If for a moment we forget about evolving Bohmian trajectories, because we are interested on the values of $S,R$ alone at each time, we could force the fluid elements to remain fixed (and go back to the Eulerian frame), or instead, we could manipluate the trajectories so as to get an adaptive grid that obeys our desires. For instance, we could force the trajectories to get more agglomerated around very spikey regions and to go away from very smooth regions, even force them to never avoid nodal regions.

To do this, note that we forced the condition that the velocity fields for the trajectories were equal to the derivatives of the phase $S$ pretty arbitrarily from the numerical standpoint. We could also have forced the velocity field to follow a certain monitor function.


\subsection*{(III.b.1) The Continuity + The Hamilton-Jacobi Equations}
\addcontentsline{toc}{subsection}{(III.b.1) The Continuity + The Hamilton-Jacobi Equations}

If we take equations \eqref{HJE} and \eqref{CE}, which are the Hamilton-Jacobi equaiton and the Continuity Equation for the fluid, and we evaluate them along the trajectory $\vec{x}(t;\vec{\xi})$, by identifying and defining:
\begin{equation}\label{v}
\dv{}{t}x_a^\xi(t)=\frac{1}{m_a}\pdv{}{x_a}S(x_a, \vec{x}_b^\xi(t),t)\Big\rvert_{x^\xi_a(t)}=:v_a(\vec{x}^\xi(t),t)
\end{equation}
we get immediately after a pair of simplifications and re-orderings that:
\begin{equation}\label{CE.L}
\dv{}{t}\rho(\vec{x}^\xi(t),t)=-\rho(\vec{x}^\xi(t),t)\sum_{a=1}^N\pdv{}{x_a} v_a(x_a, \vec{x}_b^\xi(t),t)\Big\rvert_{x_a^\xi(t)}
\end{equation}
\begin{equation}\label{HJE.L}
\dv{}{t}S(\vec{x}^\xi(t),t)=  \sum_{a=1}^N \frac{1}{2} m_a \qty(v_k(\vec{x}^\xi(t),t) )^2 - \qty(V(\vec{x}^\xi(t),t)+ Q(\vec{x}^\xi(t),t))=: \Lg(\vec{x}^\xi(t),t)
\end{equation}
Which we can write in integral form to get the value they should have along each trajectory:
\begin{equation}\label{JacPre}
\rho(\vec{x}^\xi(t),t)=e^{-\int_{t=t_0}^t \vec{\nabla}\cdot \vec{v}(\vec{x},t)\rvert_{\vec{x}^\xi(t)} dt}\rho(\vec{x}^\xi(t_0),t_0)
\end{equation}
\begin{equation}
S(\vec{x}^\xi(t),t)=\int_{t=t_0}^t \Lg(\vec{x}^\xi(t),t) dt+S(\vec{x}^\xi(t_0),t_0)
\end{equation}
Together with \eqref{v}, these three equations allow the coupled evolution of the fields along the trajectories. However, in order to evaluate the quantities $\pdv{}{x_a} v_a(x_a, \vec{x}_b^\xi(t),t)$, $\pdv{}{x_a}S(x_a, \vec{x}_b^\xi(t),t)$, $\pdv[2]{}{x_a}\rho^{1/2}(x_a, \vec{x}_b^\xi(t),t)$ of these equations, we need knowledge in other points surrounding the fluid element. To get them again, we can do one of three things:
\begin{enumerate}
\item Evolve simultaneously several fluid elements so as to numerically compute the derivatives in the unstructured grid they suppose (by fitting a function etc.).
\item Evolve more partial differential equations describing the dynamics in time for the partial derivatives, conditioned along the trajectories. This will be drafted in the following section.
\item Convert the partial deriavtives with respect to space into derivatives with respect to the labels, with the extra cost of computing the Jacobian. As the initial grid where the labels are defined is a uniform grid, typical numerical methods can be used to approximate the derivatives. 
\end{enumerate}

Once we manage to be able to evolve these fields over the trajectories, we can immediately get the full wavefunction as $\psi=Re^{\frac{i}{\hbar}S}$:
\begin{equation}
\psi(\vec{x}^\xi(t),t)=e^{-\int_{t=t_0}^t \vec{\nabla}\cdot \vec{v}(\vec{x},t)\rvert_{\vec{x}^\xi(t)} dt}e^{\frac{i}{\hbar}\int_{t=t_0}^t \Lg(\vec{x}^\xi(t),t) dt}\psi(\vec{x}^\xi(t_0),t_0)
\end{equation}
This is sometimes called the hydrodynamical wave-function propagator.
\subsubsection*{About the Fluid Jacobian}
If we analyse the evolution of the density along the trajectory at equation \eqref{JacPre}, we find that the ratio between the local density a certain fluid element $\vec{\xi}$ perceives at a time $t$ and the initial time is:
\begin{equation}
\frac{\rho(\vec{x}^\xi(t),t)}{\rho(\vec{x}^\xi(t_0),t_0)}=\frac{\rho(\vec{\xi},t)}{\rho(\vec{\xi},t_0)}=e^{-\int_{t=t_0}^t \vec{\nabla}\cdot \vec{v}(\vec{x},t)\rvert_{\vec{x}^\xi(t)} dt}
\end{equation}
This variation of the density perceived by the fluid element $\vec{\xi}$ in its local surrounding relative to the one at the beginning, can be understood in one of the following interpretations: 
\begin{itemize}
\item The number of "particles" (in the TUM sense) is diminishing (or increasing) in its surrounding, that is, there is a local divergence (or convergence) of surrounding trajectories. Meaning that given a fluid element labelled $\vec{\xi}_0$, the fluid elements that were closest to it at the time we made the labels $||\vec{\xi}_1-\vec{\xi}_2||=\varepsilon$, are now getting closer (or further) in space $||\vec{x}(\vec{\xi}_1,t)-\vec{x}(\vec{\xi}_1,t)\vec{\xi}_2||<\varepsilon$ (or $>\varepsilon$), for a small enough $\varepsilon>0$.
\item The space around the particles is contracting (or dilating). The increase or decrease of volume locally causes the density to increase or decrease relative to the first instant.
\end{itemize}
As we consider physical space to be a fixed affine manifold, the first interpretation is the most straightforward. However, if one regards the continuum of fluid elements at the initial time as the "space" or the grid, then clearly, interpretation 1 means that the continuum of fluid elements are getting more or less contracted, so the volume that the fluid occupies is dilated or contracted. In general this is the way how this quantity is referred to. If we observe the trajectories of the fluid elements as a moving variable change, $\vec{x}(\vec{\xi},t)$, as it must be a bijection for it to be a valid set of trajectories that do not cross, the function must have inverse, which in particular means that we could find the Jacobian matrix of the function $D_{\vec{\xi}}\vec{x}(\vec{\xi},t)$. The Jacobian matrix is the linear approximation of the trajectories for a local displacement in the label space $\vec{\xi}$. For the Taylor expansion around a certain trajectory of interest $\vec{\xi}^*$:
\begin{equation}
\vec{x}(\vec{\xi},t)=\vec{x}(\vec{\xi}^*,t)+D_{\vec{\xi}}\vec{x}(\vec{\xi}^*,t)\cdot (\vec{\xi}-\vec{\xi}^*)+O(||\vec{\xi}-\vec{\xi}^*||^2)
\end{equation}
For $\vec{\xi}\rightarrow \vec{\xi}^*$ we find that:
\begin{equation}
\vec{x}(\vec{\xi},t)-\vec{x}(\vec{\xi}^*,t)=D_{\vec{\xi}}\vec{x}(\vec{\xi}^*,t)\cdot (\vec{\xi}-\vec{\xi}^*)
\end{equation}
Then, the Jacobian matrix will tell us how the separation of close trajectories evolves. By the geometrical interpetation of the determinant of a linear application as the magnitude by which the unit volume of the space is scaled, we can now see that the determinant of the Jacobian will provide us the scaling factor of the separation between the trajectories, relative to the initial time. If we call this $J(\xi^*,t)=det(D_{\vec{\xi}}\vec{x}(\vec{\xi}^*,t))$, we will then realize that it must be equal to the factor we previously found:
\begin{equation}
J(\vec{\xi}^*,t)=det(D_{\vec{\xi}}\vec{x}(\vec{\xi}^*,t))=e^{-\int_{t=t_0}^t \vec{\nabla}\cdot \vec{v}(\vec{x},t)\rvert_{\vec{x}^\xi(t)} dt}
\end{equation}
An alternative, perhaps less cumbersome way to see this is by first noting that the continuity equation in the Eulerian frame \eqref{CE}, implies a local conservation of the number of particles. As integrating both sides of the equation on a configuration space volume bounded $\Omega\subset\R^N$ and applying the divergence theorem and some regularity for the density we get:
$$
\int_{\Omega}\pdv{}{t} \rho(x,t) dx=-\int_{\Omega}\sum_{k=1}^N \pdv{}{x_k}\qty( \rho(x,t)v_k(x,t) )dx=-\int_{\Omega}\vec{\nabla}\cdot \qty( \rho(x,t)\vec{v}(x,t) )dx
$$
\begin{equation}
\pdv{}{t} \int_{\Omega} \rho(x,t) dx=-\int_{\partial \Omega} \rho(x,t)\vec{v}(x,t)\cdot d\vec{S}(\vec{x})=-\int_{\partial \Omega} \rho(x,t)v_{normal}(x,t)\cdot dS
\end{equation}
The quantity in the left-hand side is the variation in time of the amount of Universes/particles inside the configuration space volume $\Omega$ (in particles/time-unit). The one in the right is the outward flux of Universes/particles across the surface. Thus, the continuity equation is equivalent to imposing that there is no source or sink for the number of particles, and thus the density. Universes cannot be destroyed nor created, only flowed thinner or denser (probability density is conserved). Once seen this, we acknowledge that it must be true that the amount of particles is conserved at al times for a fixed volume in the label space (or initial time position space):
\begin{equation}
\int_{\Omega} \rho(\vec{\xi},t_0) d\xi=\int_{\Omega} \rho(\vec{\xi},t) d\xi \quad \forall t>t_0
\end{equation}
As the trajectories are bijective, $\vec{x}(\vec{\xi},t)$ must be invertible, such that there exists a function $\vec{\xi}(\vec{x},t)$. If we evaluate it above, after noting:
\begin{equation}
\int_{\Omega} \rho(\vec{\xi}(\vec{x},t),t_0) d\vec{\xi}(\vec{x},t)=\int_{\Omega} \rho(\vec{\xi},t) d\xi \quad \forall t>t_0
\end{equation}

REYNOLDS TRANSPORT THEOREM!!!!!! Coño! Para demostrar ke la continuity eqt tb se puede sacr del flow motion xDDDDD

$$int_{x(xi,t)}: xi\in omega0 rho(x,t)dx = int_{x(xi,t0)=xi}: xi \in omega0 rho(x,to)dx$$

Y luego haces derivada temporal y uno de los lados debe ser constante etc.


\subsubsection*{(III.b.1.1) Dynamics of Partial Derivatives}
\addcontentsline{toc}{subsubsection}{(III.b.1.1) Dynamics of Partial Derivatives}

\subsubsection*{(III.b.1.2) Partial Derivatives relative to the Labels}
\addcontentsline{toc}{subsubsection}{(III.b.1.2) Partial Derivatives relative to the Labels}

Esto es como revertir el Lagrangian into an Eulerian! Ze el grid de los labels es siempre estacionario para las trayectorias! 

Claro, será necesario computar el JAcobiano de la densidad para saber x relativo a las xi, pero se supone que el jacobiano lo puedes sacar de las trayectorias! aSike si eres capaz de mover las trajs pa sacar el jacobiano el problema de las derivadas en y se solucionaria porke ahora son derivadas en chi!

\subsection*{(III.b.1.3) Adaptive Grid Equations}
\addcontentsline{toc}{subsubsection}{(III.b.1.3) Adaptive Grid Equations}

If for a moment we forget about evolving Bohmian trajectories, because we are interested on the values of $S,R$ alone at each time, we could force the fluid elements to remain fixed (and go back to the Eulerian frame), or instead, we could manipluate the trajectories so as to get an adaptive grid that obeys our desires. For instance, we could force the trajectories to get more agglomerated around very spikey regions and to go away from very smooth regions, even force them to never avoid nodal regions.

To do this, note that we forced the condition that the velocity fields for the trajectories were equal to the derivatives of the phase $S$ pretty arbitrarily from the numerical standpoint. We could also have forced the velocity field to follow a certain monitor function.


\subsection*{(III.b.2) The Bohmian-Newton's Second Law}
\addcontentsline{toc}{subsection}{(III.b.2) The Bohmian-Newton's Second Law}

If we take the Hamilton-Jacobi Equation in the fully Eulerian frame \eqref{HJE} and do the partial derivative $\pdv{}{x_k}$ at both sides, assuming again \eqref{ImpBohm}, we arrive assuming we can swap cross derivatives by Schwartz's Law (enough regularity for $S$) to:
\begin{equation}
\pdv{}{t}\pdv{}{x_k}S(x,t)+\sum_{j=1}^N\pdv{}{x_j}\pdv{}{x_k}S(x,t)\cdot v_j(x,t)=-\pdv{}{x_k}\qty(V(x,t)+Q(x,t))
\end{equation}
Which if we evaluate in the Lagrangian frame $\vec{x}(\vec{\xi},t)$, we immediately find (assuming \eqref{v}):
\begin{equation}
m_k\dv{}{t}v_k(\vec{x}^\xi(t),t)=m_k\dv[2]{}{t}x^\xi_k(t)=-\pdv{}{x_k}\qty[V(\vec{x}^\xi(t),t)+Q(\vec{x}^\xi(t),t) ]
\end{equation}
Which is Newton's Second Law for the fluid elements or Bohmian trajectories. This equation can be evolved coupled with \eqref{CE.L} in order to solve the same quantum problem. In this representation however, we will require to compute the spatial derivative of the quantum potential $Q$, which was already a problemtic term due to the $\pdv[2]{}{x_j}R(x,t)$ it contains. Therefore, for numerical purposes it is more stable to use the equations of motion of III.b.1.


\subsection*{(III.c) Basis Set Expansion}
\addcontentsline{toc}{subsection}{(III.c) Basis Set Expansion}
\subsection*{(III.d) Dynamic Equations for Partial Differentials}
\addcontentsline{toc}{subsection}{(III.d) Dynamic Equations for Partial Differentials}

\section*{II . Half Lagrangian Half Eulerian Equations:\\ Conditional Wave-Functions}
\addcontentsline{toc}{section}{II . Half Lagrangian Half Eulerian Equations: Conditional Wave-Functions}
In this section, we will explore an intermediate approach between the fully Lagrangian and Eulerian frames. For this, we will consider that part of the system behaves in a Lagrangian frame, while the other part behaves in an Eulerian frame. The degrees of freedom described on an Eulerian frame will be denoted by $\vec{x}=(x_1,...,x_m)$ while the rest of degrees of freedom $\vec{y}=(x_{m+1},...,x_N)$ will be described by a Lagrangian frame. If we need to refer to both kinds of variables at once, we will use $\vec{\x}=(\vec{x}, \vec{y})=(x_1,...,x_N)$.

In general, given we parametrize the fluid elements with labels $\vec{\xi}=(\vec{\xi}_x,\vec{\xi}_y)\in\R^N$ referring to their initial position $\vec{\xi}:=\vec{\x}(t_0, \vec{\xi})$, we define the set of trajectories of the continuum as $\vec{\x}(t;\vec{\xi}) \equiv \vec{x}^\xi(t)$. At the first parts of this section however, we will only treat in the Lagrangian frame the degrees of freedom in $\vec{y}$. We will then denote by $\vec{y}^\xi(t)=(x_{m+1}^\xi(t),...,x_N^\xi(t))$ the set of trajectories for the subsystem $\vec{y}$.

As such, all the quantities of the fields we will describe here will be of the shape $f(\vec{x}, \vec{y}^\xi(t),t)=f(\vec{x}, \vec{\xi},t)$. We will call these the {\bf conditional} fields, as each trajectory of the Lagrangian degrees of freedom will imply an $m$ dimensional "slice" of the $N$ dimensional field $f(\vec{\x},t)$. It is this why we will sometimes call the degrees of freedom $\vec{y}$ "transversal sections". You can see in Figure \ref{fig:transv} some representations of conditional fields.


\subsection*{(II.a.1) The Schrödinger Equation: Kinetic and Advective}
\addcontentsline{toc}{subsection}{(II.a.1) The Schrödinger Equation: Kinetic and Advective}
If we evaluate $\vec{y}(t,\vec{\xi})$ in the Schrödinger Equation, leaving $\vec{x}$ in the Eulerian frame:
\begin{equation}
i\hbar \pdv{}{t}\psi(\vec{x}, \vec{y}^\xi(t),t)=-\sum_{j=1}^m \frac{\hbar^2}{2m_j}\pdv[2]{}{x_j} \psi(\vec{x}, \vec{y}^\xi(t),t) + U(\vec{x}, \vec{y}^\xi(t),t)\psi(\vec{x}, \vec{y}^\xi(t),t) -\sum_{j=m+1}^N \frac{\hbar^2}{2m_j}\pdv[2]{}{x_j} \psi(\vec{x}, \vec{y},t)\Big\rvert_{\vec{y}^\xi(t)}
\end{equation}
Using that by the chain rule:
\begin{equation}
\dv{}{t}\psi(\vec{x}, \vec{y}^\xi(t),t)=\pdv{}{t}\psi(\vec{x}, \vec{y}^\xi(t),t)+\sum_{j=m+1}^N \pdv{}{x_j}\psi(\vec{x}, \vec{y},t)\Big\rvert_{\vec{y}^\xi(t)}\cdot \dv{}{t}x^\xi_j(t)
\end{equation}
We get:
\begin{equation}\label{KinAdvEL}
i\hbar \dv{}{t}\psi(\vec{x}, \vec{y}^\xi(t),t)=\qty[-\sum_{j=1}^m \frac{\hbar^2}{2m_j}\pdv[2]{}{x_j} + U(\vec{x}, \vec{y}^\xi(t),t)]\psi(\vec{x}, \vec{y}^\xi(t),t)+
\end{equation}
$$
+i\hbar\sum_{j=m+1}^N \pdv{}{x_j}\psi(\vec{x}, \vec{y},t)\Big\rvert_{\vec{y}^\xi(t)}\cdot \dv{}{t}x^\xi_j(t) -\sum_{j=m+1}^N \frac{\hbar^2}{2m_j}\pdv[2]{}{x_j} \psi(\vec{x}, \vec{y},t)\Big\rvert_{\vec{y}^\xi(t)}
$$
Which if we define the so called Kinetic and Advective correlation potentials:
\begin{equation}\label{KinEL}
K(\vec{x}, \vec{y}^\xi(t),t)=-\sum_{j=m+1}^N \frac{\hbar^2}{2m_j}\pdv[2]{}{x_j} \psi(\vec{x}, \vec{y},t)\Big\rvert_{\vec{y}^\xi(t)}
\end{equation}
\begin{equation}\label{AdvEL}
A(\vec{x}, \vec{y}^\xi(t),t)=i\hbar\sum_{j=m+1}^N \pdv{}{x_j}\psi(\vec{x}, \vec{y},t)\Big\rvert_{\vec{y}^\xi(t)}\cdot \dv{}{t}x^\xi_j(t)
\end{equation}
we can see that equation \eqref{KinAdvEL} ruling the motion of the so called conditional wavefunctions $\psi(\vec{x}, \vec{y}^\xi(t),t)$ is almost a Schrödinger Equation for a system of $m$ dimensions (the Eulerian piece). The difference is that we now have two additional affine terms that actually depend on derivatives of the full wavefunction in the axes that we are considering in the Lagrangian frame. Clearly, if we want to compute these, we do not have enough with the conditonal wavefunction for a certain trajectory. 

The thing would be better if we also knew the value of the wavefunction at other points that are not the single $\vec{y}^\xi(t)$. Much in the same way that we did with the purely Lagrangian frame, for the piece considered with point-like trajectories, we will need to have several representatives to be able to approximate the derivatives in those directions.

Before further digging into the significance of this approach and how we could deal with this equation, let us review what we really mean by a conditional wavefunction.

\subsubsection*{What do we mean by a Conditional Wavefunction?}
\addcontentsline{toc}{subsubsection}{What do we mean by a Conditional Wavefunction?}
First of all, note that in the development of equation \eqref{KinAdvEL}, we considered no rule for the evolution of the trajectory $\vec{y}^\xi(t)$. In fact, we could impose them to be still or move in random directions, and if we were able to evolve equation \eqref{KinAdvEL}, the values of the conditional wavefunction (CWF) $\psi(\vec{x}, \vec{y}^\xi(t),t)$ would still be exactly correct. It is this freedom of choosing the motion of the Lagrangian trajectories that will map the space for the degrees of freedom $\vec{x}^\xi_b$, that will allow us to build adaptive grids in a coming section.

However, if we evolved the trajectories with an arbitrary law of motion, yes, we would obtain correct values for the wavefunction over the spots where they move, yet, the trajectories themselves would not provide us much physical insight. This will be true unless their motion depends on the state of the wavefunction, the fluid. We will see in the adaptive grid equations, that in fact we will be able to get information about a custom monitor function just from the movement of the trajectories. In particular, if we make the trajectories follow the flow lines of the fluid or Bohmian trajectories, we will get information of big interest. Why? Because these will be the trajectories that each particular experiment (each particular Universe) will take. Of course, experimental results will only produce stochastical results sampled from the density of the fluid, due to our lack of perfect knowledge of the position of all the particles in the Universe. However, this will allow us for instance to know about the past of a certain observation, about the dispersion or concentration of probability density etc.

We will now notice though that saying the Lagrangian part of the system $\vec{y}^\xi$ (the one that will be treated as an ensemble of particles) will follow probability density flow lines or Bohmian trajectories, is not that well defined.



\subsubsection*{(a) If we only consider that there is trajectory in $\vec{y}$}

If we simply define the velocity field for $\vec{y}^\xi(t)$ to be:
\begin{equation}
\dv{}{t}x_j^\xi(t)=\frac{1}{m_j}\pdv{S(\vec{x}, \vec{y})}{x_j}=\frac{\hbar^2}{m_j}\Im\qty( \psi^{-1} \pdv{\psi}{x_j})\Big\rvert_{\vec{y}^\xi(t)}
\end{equation}
We see that for each value of $\vec{x}$ the Eulerian degrees, we have a different velocity with which to move the trajectory in $\vec{y}$. In reality these velocities in $y$ for each $x$ are due to the fact that each $x$ can be seen as a fluid element if we make the degrees $x$ also Lagrangian. However, in this first view, we assume there is only trajectory in $y$. If we took a certain initial $y^\xi(t_0)=\xi_y$ we would find in the first time iteration of the CWF that each point $x$ is moved by a different velocity in $y$. Following these trajectories, what we would achieve as is shown in Figure \ref{fig:only_y} is that each $x$ would always maintain a single value of the wavefunction, because the $x$ points are fixed.Then the evolution of the CWF would describe the evolution of a function graph, an $m$ dimensional manifold with a single complex wavefunction value per each point of the manifold.

% FIGURE

If we tried to understand what really is happening from global Bohmian trajectory's perspective, it is certain that we are not evolving Bohmian trajectories of the full system. It is as if trajectories were restricted to be moved in $x$. It seems to be the price to pay for Eulerian grids.

A good thing about this would be that if we evolved several CWF-s for different single initial positions in $y^\xi(t_0)=\xi_y$, the x points would always be aligned between the CWF-s, with the computational advantage that computing derivatives or interpolating in $y$ would be simplified, even if the $y$ grid for each $x$ would get unstructured with time. See Figure \ref{fig:only_y_grid}

% FIGURE

Thus the trajectories we obtain for $y^\xi$ are certainly not Bohmian trajectories, since we are forcing that $x$ positions are still. It would be as if we fixed the equations of motion for the trajectories:
\begin{equation}
\begin{cases}
\vec{x}^\xi(t)=\vec{\xi}_x \quad \forall t\geq t_0\\
\vec{y}^\xi(t)=\vec{\xi}_y+\int_{t_0}^t \vec{v}_y(\vec{x}, \vec{y}(t),t)dt\\ \\
v_j(\vec{x}, \vec{y},t)=\frac{1}{m_j}\pdv{S(\vec{x}, \vec{y},t)}{x_j}\quad j\in\{m+1,...,N \}
\end{cases}
\end{equation}

\subsubsection*{(b) If we consider that there is trajectory in $\vec{\x}$}
What if we allowed $x$ positions to move?
\begin{equation}
\begin{cases}
\vec{x}^\xi(t)=\vec{\xi}_x +\int_{t_0}^t \vec{v}_x(\vec{x}(t), \vec{y}(t),t)dt\quad \forall t\geq t_0\\
\vec{y}^\xi(t)=\vec{\xi}_y+\int_{t_0}^t \vec{v}_y(\vec{x}(t), \vec{y}(t),t)dt\\ \\
v_j(\vec{x}, \vec{y},t)=\frac{1}{m_j}\pdv{S(\vec{x}, \vec{y},t)}{x_j}\quad j\in\{1,...,N \}
\end{cases}
\end{equation}

Well, then the CWF would evolve at each point following the fluid flow and it would turn into an unstructured grid in all axes. We would now have a varying number of values of wavefunction per $x$ as a function of time. It would still be an $m$ dimensional manifold, meaning the parametrization given by $\vec{\xi}_x$ would still be valid, and we would still have one point in the wavefunction per each  $\vec{\xi}_x$ in the initial manifold. However, as said, a certain $x$ would now be possible to have multiple wavefunction evaluations. For all practical means we would have falled back to the fully Lagrangian frame! See Figure \ref{fig:fallback}.

\subsubsection*{(c) If we consider that each CWF moves along a single trajectory}
In (a) we found that part of the system was really in the Eulerian frame and part was really in the Lagrangian one. However, the trajectories we evolved were not useful. In (b) we found that having really Bohmian trajectories move the CWF resulted in the fully Lagrangian frame. However, notice that this was only because we allowed each point of the CWF to move along the flow lines. What if we forced all the points of the CWF to move together along a single Bohmian trajectory, a single flow line?

Notice one of the most interesting properties of the CWF-s: a single slice, a single CWF is enough to define the velocity field in $x$. The (Bohmian) velocity field is due to a derivative in the $x$ axis and we are treating this axis in the Eulerian frame (thus we know its values along all $x$ for this $y^\xi(t)$). But again, if we simply define the velocity field in $x$ for each and every $x$ and move them accordingly we will loose the Eulerian grid. What we can do is the following: define a trajectory in $x$ for the CWF. That is, choose one $\xi_x$ at random appart from the $\xi_y$ we already chose and use the same CWF to propagate $\vec{x}^\xi(t)$ it in time. Then use this $x$ position to evaluate the velocity field in $y$ for the initial $\xi_y$. This way, at all times we will only have a single trajectory to track, and in fact it will be a Bohmian trajectory by definition. That is, the CWF will move in space preserving its affine shape and it will not get branched in several $y$ trajectories. See Figure \ref{fig:singleTraj}.

% FIGURE

\subsubsection*{(d) Two Coupled Conditional Wavefunctions?}

We now accept that in order to preserve a half Eulerian half Lagrangian frame feasible and still obtain Bohmian trajectories, we need to evolve only one Bohmian trajectory per CWF. We also know that the CWF is enough to get the velocity field in its Eulerian degrees of freedom but not for the Lagrangian degrees of freedom. For those degrees of freedom we have the same problem that we had in the fully Lagrangian: need several trajectories, thus several CWF-s, to approximate the derivatives. But, if we will only evolve a single trajectory per CWF: couldn't we evolve a coupled CWF with the Eulerian degrees being the Lagrangian of the first and vice versa? This way, we could get the velocity field that guides the joint trajectory evolved only with th eCWF pair! No need for interpolation! Well, not really.

It is true that if we choose a certain initial point for the trajectory of a pair of CWF-s $\vec{\xi}=(\vec{\xi}_x, \vec{\xi}_y)$ and define the CWF-s as:
\begin{equation}
\psi_x^\xi(\vec{x},t):=\psi(\vec{x}, \vec{y}^\xi(t),t) \text{  and  } \psi_y^\xi(\vec{y},t):=\psi(\vec{y}, \vec{x}^\xi(t),t)
\end{equation}
then the trajectory $(\vec{x}^\xi(t), \vec{y} xi(t))$ will be entirely determined by the CWF pair by:
\begin{equation}
\begin{cases}
\vec{x}^\xi(t)=\vec{\xi}_x +\int_{t_0}^t \vec{v}^x(\vec{x}^\xi(t), \vec{y} ^\xi(t),t)dt\quad \forall t\geq t_0\\
\vec{y}^\xi(t)=\vec{\xi}_y+\int_{t_0}^t \vec{v}^y(\vec{x}^\xi(t), \vec{y}^\xi(t),t)dt\\ \\
v_j^k(\vec{x}, \vec{y},t)=\frac{1}{m_j}\pdv{S^x(\vec{x}, \vec{y},t)}{x_j}= \frac{\hbar^2}{m_j}\Im\qty( \psi_k^{-1} \pdv{\psi_k}{x_j})\Big\rvert_{\vec{y}^\xi(t)} \quad k\in\{ x,y\}
\end{cases}
\end{equation}

However, looking back at equation \eqref{AdvEL}, guiding the time evolution of the CWF-s, we notice that we will still lack the knowledge of the derivatives in $y$ for all $x$ except for the one where the trajectory currently is. Then once again, the solution will be to evolve several trajectories using their coupled CWF-s in paralell for each time iteration and use all of them to approximate the derivatives in their Lagrangian axes, just like we did in the previous Section.

\subsubsection*{(e) n Coupled Conditional Wavefunctions}
In fact, we could do the previous trick by evolving not two, but $n$, up to $N$ different CWF-s, where each could even have different Eulerian dimensionalities, and where each has as Eulerian degrees a different degree of freedom. Such that If $n=N$ then we would have a 1D Eulerian CWF per axis, if $n=N/3$ we could have one 3D Eulerian CWF per 3D physical space particle, or even stranger combinations like the one shown in Figure \ref{fig:strangeSlicing}. All of them would allow the proper time evolution of a single Bohmian trajectory

%FIGURE

\subsubsection*{What is the advantage of the Half Lagrangian-Half Eulerian approach?}
The advantage is that we might be able to get the best of both worlds: the Eulerian and the Lagrangian.

\begin{itemize}
\item The more dimensional the CWF's Eulerian degrees are, the smaller the space for sampling the trajectories of the Lagrangian part will be. This means that the less CWF-s will be required to reproduce the full wavefunction or to achieve a certain precision in the derivatives in Lagrangian directions. However, the more computationally complex will be the CWF-s to evolve, so the problem gets more complex to be parallelized. In the limit, of $m=N$ we would recover the linear Schrödinger equation in the Eulerian frame, which scales exponentially in time with dimensions.

\item The less dimensional the CWF's Eulerian degrees, the more trajectories we will need to reconstruct the full wavefunction and to get the derivatives in the Lagrangian directions, in fact, presumably exponentially more. However, the CWF-s will be simpler to evolve and as the trajectories can (and must) be computed coupled but in parallel, the more of exponential complexity we will transfer to parallel threads. In the limit of $m=0$, we recover the fully Lagrangian frame, where we achieve the apparently highest parallelizability of the Quantum many body problem.

\item Unlike the fully Lagrangian frame, in this case, each Bohmian trajectory is accompanied by lots of wavefunction points (not just one) and therefore, interpolating the full wavefunction is way simpler, by for instance a nearest neighbour approach. Each CWF is a full $mD$ affine hyperplane of values for the wavefunction.

\item The grid is preserved in a structured manner at all times for the Eulerian degrees, while it gets unstructured for the Lagrangian axes. However, if coupled CWF-s are present, there are parts of the Lagrangian axes for one CWF that are Eulerian for the other, so there is always track of the wavefunction in all the extent of the simulation domain. The intersting thing is that the points in x of different trajectory CWF-s are always aligned, so the numerical derivatives in y or interpolations are ways impler than in a fully unstructured grid.

\item {\bf Fast and Slow Degrees of Freedom}: CWF-s provide a natural way to treat the slow degrees of freedom separated form the fast ones, like an atom nuclei from the orbiting electrons. It is common in molecular dynamics algorithm to consider classical trajectories for parts of the quantum system and quantum wavefunctions for other parts. We will see how we could do this using CWF-s in the Bohmian Newton's Equation.
\end{itemize}

% FALTAN COMENTS EN SE QTM Y AKI DE LA COMPLEJIDAD Y PARALELIZABILIDAD. Y LA TRANSFERENCIA ENTRE INTEGRAL EIGENSTATE COMPUTATION Y ESTO Y LO OTRO KIZAS ESO LO PODRIA HACER AL FINAL MEJOR SUPONGO.


\subsection*{(II.b.1) The Continuity + The Hamilton-Jacobi Equations}
If we evaluate the trajectory $\vec{y}(t;\vec{\xi})$ in the Continuity equation \eqref{CE}, we obtain:
\begin{equation}
\dv{}{t}\rho(\vec{x},\vec{y}^\xi(t),t)=-\sum_{j=1}^m\pdv{}{x_j}\qty(\rho(\vec{x}, \vec{y}^\xi(t), t) \frac{1}{m_j}\pdv{S(\vec{x}, \vec{y}^\xi(t), t)}{x_j}) - \rho(\vec{x}, \vec{y}^\xi(t),t) \sum_{j=m+1}^N \frac{1}{m_j}\pdv[2]{S(\vec{x}, \vec{y}, t)}{x_j}\Big\rvert_{\vec{x}^\xi(t)}
\end{equation}
Which is a continuity equation for $\rho(\vec{x},\vec{y}^\xi(t),t)$ with a source term $- \rho(\vec{x}, \vec{y}^\xi(t),t) \sum_{j=m+1}^N \frac{1}{m_j}\pdv[2]{S(\vec{x}, \vec{y}, t)}{x_j}\Big\rvert_{\vec{x}^\xi(t)}$ that drains or injects density as a function of the sign of the gradient of the velocity field in the Lagrangian axes (the contraction or dilation of the volume element, the determinant of the Jacobian of the density for the Lagrangian degrees). This is why the time evolution of the CWF-s is non-unitary. This was already clear in the fully Lagrangian scheme, because the density of the fluid was not a conserved quantity along the trajectories, meaning that the density that each fluid element perceives can vary in time. We actually found that the amount of density a fluid element perceived changed in time with the dilatation and contraction of the trajectory bundle, as given by the Jacobian of the mesh.

Evaluating the trajectory for the Lagrangian axes in the Hamilton-Jacobi equation \eqref{HJE} we get the ugly equation:
\begin{equation}
-\dv{}{t}S(\vec{x}, \vec{y}^\xi(t),t)=\sum_{j=1}^m\frac{1}{2m_j}\qty(\pdv{S(\vec{x},\vec{y}^\xi(t),t)}{x_j})^2-\sum_{j=m+1}^N\frac{1}{2m_j}\qty(\pdv{S(\vec{x},\vec{y}^\xi(t),t)}{x_j})^2+U(\vec{x},\vec{y}^\xi(t),t)+
\end{equation}
$$
-\sum_{j=1}^m \frac{\hbar^2}{2m_jR(\vec{x},\vec{y}^\xi(t),t)}\pdv[2]{R(\vec{x},\vec{y}^\xi(t),t)}{x_j}-\sum_{j=m+1}^N \frac{\hbar^2}{2m_jR(\vec{x},\vec{y}^\xi(t),t)}\pdv[2]{R(\vec{x},\vec{y},t)}{x_j}\Big\rvert_{\vec{x}^\xi(t)}
$$
Which is the Hamilton-Jacobi equation for the CWF, except that there are two terms extracting energy: the kinetic energy of Lagrangian frame and the quantum potential contribution due to the agglomeration in that axis.

%\subsection*{(II.b.2) The Bohmian-Newton's Second Law}
%Evaluating the trajectory $\vec{y}(t;\vec{\xi})$ in the Continuity equation \eqref{CE}, we obtain:

\subsection*{(II.a.2) The Schrödinger Equation: G and J Correlations}
\addcontentsline{toc}{subsection}{(II.a.2) The Schrödinger Equation: G and J Correlations}
In this section we will develop a mix between the Continuity Equation and Hamilton-Jacobi Equation of the previous subsection and the Schrödinger Equation. In fact, the Lagrangian degrees of freedom will be described using the density and action, while the Eulerian part will be described by a wavefunction. 

Following the development in Chp.1 V 6 of \cite{JordiXO}: We can try to find a {\bf linear} Schrödinger like equation for the CWF-s employing the following "trick". An arbitrary non-zero single valued complex function $f(x, t):\R^2 \rightarrow \C$ can be imposed to be the solution of a 1D Schrödinger equation:\vspace{-0.3cm}
$$
i \hbar \pdv{f(x,t)}{t} = -\frac{\hbar^2}{2 m}\pdv[2]{f(x,t)}{x} + W(x,t) f(x,t)
$$
if the potential term $W(x, t)$ is defined as:\vspace{-0.3cm}
$$
W(x, t) := \qty(i \hbar \pdv{f(x,t)}{t} +\frac{\hbar^2}{2 m}\pdv[2]{f(x,t)}{x} ) \frac{1}{f(x,t)}\vspace{-0.3cm}
$$
The proof is immediate. An observation that we must note is that for an arbitrary $f(x,t)$, the potential $W(x,t)$ can be complex as well! Which is not the case in the usual Schrödinger Equation. But we already assumed that \ref{K} and \ref{A} could be complex too, so that's okay. \vspace{-0.3cm}\\

Then, using this for the CWF-s $\psi_a^\beta(x_a,t)$, we will obtain a third alternative shape for (for \hyperref[CWF.SE]{CWF.SE}).

We must first evaluate the CWF in polar form $\psi^\beta_a(x_a,t)=\p(x_a,t) e^{i\z(x_a,t) / \hbar}$ in the expression for $W(x_a,t)$:
$$
W(x_a, t) = \qty(i \hbar \pdv{\psi_a^\beta(x_a,t)}{t} +\frac{\hbar^2}{2 m_a}\pdv[2]{\psi_a^\beta(x_a,t)}{x_a} ) \frac{1}{\psi_a^\beta(x_a,t)} = \qty(i \hbar \pdv{\qty(\p e^{i\z/\hbar})}{t} +\frac{\hbar^2}{2 m_a}\pdv[2]{\qty(\p e^{i\z/\hbar})}{x_a} ) \frac{1}{\p e^{i\z/\hbar}}
$$
using the Leibniz derivation rule several times and an inverse chain rule, rearranging we arrive at:
$$
W(x_a, t)=-\frac{1}{2m_a} \qty( \qty(\pdv{\z_a}{x_a})^2 -\frac{\hbar^2}{\p_a}\pdv[2]{\p_a}{x_a})\ -\pdv{\z_a}{t}+ \ i \ \frac{\hbar}{\p_a^2} \qty(\pdv{\p_a^2}{t}+\pdv{}{x_a}\qty(\frac{\p_a^2}{m_a}\pdv{\z_a}{x_a}))
$$
%Separating the real and imaginary parts:
%$$
%\begin{cases}
%\mathbb{R}e\{W(x_a,t)\}=-\pdv{\z_a(x_a,t)}{t}-\frac{1}{2m_a} \qty( \qty(\pdv{\z_a(x_a,t)}{x_a})^2 -\frac{\hbar^2}{\p_a(x_a,t)}\pdv[2]{\p_a(x_a,t)}{x_a})\vspace{-0.3cm} \\ \\
%\mathbb{I}m\{W(x_a, t)\} = \frac{\hbar}{\p_a(x_a,t)^2} \qty(\pdv{\p_a^2}{t}+\pdv{}{x_a}\qty(\frac{\p_a^2}{m_a}\pdv{\z_a}{x_a}))\vspace{-0.1cm}
%\end{cases}
%$$
%Where we can recognize the \ref{QHJE} for a single particle in 1D. As such, the real part of W is simply the scalar real potential, then the Hamiltonian, followed by the kinetic energy and the quantum potential of a 1D particle.
%Which is clearly a modified particle conservation equation \ref{CE}. Note how if  $Im\{W(x_a, t)\} =0$ then we get the common continuity equation, which would mean that probability is conserved, and the solution $\Phi(x_a,t)$ would preserve its norm at all times (the conditional $r_a^2$ would integrate a same norm at all times in its spatial dimension $x_a$). Nonetheless, if $Im\{W(x_a, t)\} \neq 0$, then particles/probability are NOT conserved, and their source or sink will be quantified by $\frac{2 \_a^2}{\hbar} Im\{W(x_a, t)\}$. Therefore, the norm of $\psi_a^\beta(x_a,t)$ will not need to be preserved in the time evolution. \vspace{-0.3cm}\\

If $W$ has that shape, $\psi_a^\beta(x_a,t)$ will be the solution of the differential equation:\vspace{-0.2cm}
$$
i \hbar \pdv{\psi_a^\beta(x_a,t)}{t} = -\frac{\hbar^2}{2 m}\pdv[2]{\psi_a^\beta(x_a,t)}{x_a} + W(x_a,t) \psi_a^\beta(x_a,t)\vspace{-0.2cm}
$$
which if $\mathbb{I}m\{W\} =0$ would look like an actual single particle \ref{TDSE}. However, $W$ depends on parts of the CWF itself, so the differential equation is indeed non-linear even in that case.\\

We can further develop the expression of $W$ using the conditional definition of $\psi_a^\beta$. Note that $\psi_a^\beta ( x_a, t) = \Psi(x_a,t; \ \vec{x}_b^\beta(t))$ and thus $\z(x_a,t)=S(x_a, \vec{x}_b^\beta(t),t)$ and $\p (x_a,t)=R(x_a, \vec{x}_a^\beta (t),t)$, where we have that the full wavefunction in polar form is $\Psi(\vec{x},t)=R(\vec{x},t)e^{iS(\vec{x},t)/\hbar}$. Carefully evaluating them in $W$ and applying the chain rule, the real part of $W(x_a,t)=\W(x_a,t;\ \vec{x}_b^\beta(t))$ yields:
$$
\R e\{W(x_a,t)\}=\R e\{\W(x_a,t; \vec{x}_b^\beta(t))\}=\vspace{-0.12cm}
$$
$$
-\frac{1}{2m_a} \qty(\pdv{S(x_a,t;\vec{x}_b^\beta(t))}{x_a})^2 +\frac{\hbar^2}{2m_aR(x_a,t;\ \vec{x}_b^\beta(t))}\pdv[2]{R(x_a, t; \ \vec{x}_b^\beta(t))}{x_a}-\dv{S(x_a,t,\vec{x}_b^\beta(t))}{t}=\vspace{-0.1cm}
$$
$$
-\frac{1}{2m_a} \qty(\pdv{\z_a(x_a,t)}{x_a})^2 +\frac{\hbar^2}{2m_a\p_a(x_a,t)}\pdv[2]{\p_a(x_a, t)}{x_a}- \ \qty(\pdv{S(x_a,t,\vec{x}_b)}{t}\Big\rvert_{\vec{x}_b^\beta(t)}+\sum_{k=1;\ k\neq a}^n \pdv{S(x_a,t,\vec{x}_b)}{x_k}\Big\rvert_{x_k^\beta(t)} \dv{x_k^\beta(t)}{t})\vspace{-0.12cm}
$$

Note how the only terms introducing some coupling with the rest of particles are the last two. They are the source of the {\bf entanglement}, {\bf exchange} and {\bf correlations} with the rest of the dimensions. Now, knowing that the full wave-function follows the \ref{TDSE} and thus the Quantum Hamilton-Jacobi Equation \cite{nireTFGie}, we can evaluate it for $\pdv{S(x_a,t,\vec{x}_b)}{x_k}$ in the equation above:\vspace{-0.3cm}
$$
\R e\{W(x_a,t)\}=\ Re\{\W(x_a,t; \vec{x}_b^\beta(t))\}=\vspace{-0.14cm}
$$
\begin{equation*}
\begin{split}
-\frac{1}{2m_a} \qty(\pdv{\z_a(x_a,t)}{x_a})^2 +\frac{\hbar^2}{2m_a\p_a(x_a,t)}\pdv[2]{\p_a(x_a, t)}{x_a}- \ \sum_{k=1;\ k\neq a}^n \qty( \pdv{S(x_a,t,\vec{x}_b)}{x_k}\Big\rvert_{x_k^\beta(t)} \dv{x_k^\beta(t)}{t})\ + \\ \ +\sum_{k=1}^n \qty[\frac{1}{2m_k} \qty(\pdv{S}{x_k}\Big\rvert_{\vec{x}_b^\beta(t)})^2 -\frac{\hbar^2}{2m_kR}\pdv[2]{R}{x_k}\Big\rvert_{\vec{x}_b^\beta(t)} ] - U(x_a,t; \vec{x}_b^\beta(t))
\end{split}
\end{equation*}
Observe that in the last sum, the $k=a$ term is equal to the two initial terms, which cancel each other out and we are left with the final expression:\vspace{-0.3cm}\label{ReW}
\begin{equation*}
\R e\{\W(x_a,t; \vec{x}_b^\beta(t))\}=\sum_{k=1;\ k \neq a}^n \qty[\frac{1}{2m_k} \qty(\pdv{S}{x_k}\Big\rvert_{\vec{x}_b^\beta(t)})^2 -\frac{\hbar^2}{2m_kR}\pdv[2]{R}{x_k}\Big\rvert_{\vec{x}_b^\beta(t)} -\pdv{S}{x_k}\Big\rvert_{x_k^\beta(t)} \dv{x_k^\beta(t)}{t} ] + V(x_a,t; \vec{x}_b^\beta(t))
\end{equation*}
We now have defined $\R e(W)$ without using $\psi_a^\beta$ in the same definition (necessary if we want to use the Schrödinger like equation computationally), at the cost of introducing the full wave-function to it. We see that this real part of $W$ is composed by the classical conditional potential energy $U$ introducing geometric constrictions between the coordinates and an additional part that stands for the quantum correlation with the rest of the system. We will call this the potential $G_a$:
\begin{equation}\label{G}\tag{G}
G_a(x_a,t;\ \vec{x}_b^\beta(t)):=  \sum_{k=1;\ k \neq a}^n \qty[\frac{1}{2m_k} \qty(\pdv{S}{x_k}\Big\rvert_{\vec{x}_b^\beta(t)})^2 -\frac{\hbar^2}{2m_kR}\pdv[2]{R}{x_k}\Big\rvert_{\vec{x}_b^\beta(t)} -\pdv{S}{x_k}\Big\rvert_{x_k^\beta(t)} \dv{x_k^\beta(t)}{t} ]
\end{equation}
Performing the same development for the imaginary part of $W$, that is, evaluating the definition of \ref{CWF} in $\mathbb{I}m\{W(x_a,t)\}$ and applying the chain rule:
$$
Im\{W(x_a,t)\}=Im\{\W(x_a,t; \vec{x}_b^\beta(t))\}=\vspace{-0.1cm}
$$
$$
\frac{\hbar}{2R^2}\Big\rvert_{\vec{x}_b^\beta(t)} \qty( \pdv{R(x_a,t;\vec{x}_b^\beta(t))^2}{t} + \pdv{}{x_a} \qty(\frac{R^2}{m_a} \pdv{S(x_a,t; \vec{x}_b^\beta(t))}{x_a} ))=\vspace{-0.1cm}
$$
$$
\frac{\hbar}{2R^2}\Big\rvert_{\vec{x}_b^\beta(t)} \qty( \pdv{R(x_a,t, \vec{x}_b)^2}{t}\Big\rvert_{\vec{x}_b^\beta(t)} + \sum_{k=1;\ k \neq a}^n \pdv{R^2}{x_k}\Big\rvert_{\vec{x}_b^\beta(t)} \dv{x_k^\beta(t)}{t} + \pdv{}{x_a} \qty(\frac{R^2}{m_a} \pdv{S(x_a,t; \vec{x}_b^\beta(t))}{x_a} ) )
$$
As the whole wave-function follows the \ref{TDSE}, a N-particle continuity equation must be followed \cite{nireTFGie}, which can be evaluated at $\pdv{R(x_a,t, \vec{x}_b)^2}{t}$. Noting there is a cancellation of the $k=a$ term (as happened with the real case), we arrive at an expression independent of $\psi_a^\beta$ for the imaginary part. We will define the potential energy term $J_a(x_a,t; \vec{x}_b^\beta(t)):=\mathbb{I}m\{\W(x_a,t; \vec{x}_b^\beta(t))\}$.
\begin{equation}\label{J}\tag{J}
J_a(x_a,t; \vec{x}_b^\beta(t)):= \frac{\hbar}{2R^2}\Big\rvert_{\vec{x}_b^\beta(t)} \sum_{k=1;\ k \neq a}^n \qty[ \pdv{R^2}{x_k}\Big\rvert_{\vec{x}_b^\beta(t)} \dv{x_k^\beta(t)}{t} - \frac{1}{m_k} \pdv{}{x_k} \qty(R^2 \pdv{S}{x_k} )\Big\rvert_{\vec{x}_b^\beta(t)} ]
\end{equation}

With all, we have that the complex potential is decomposed in the following potential terms:
$$
W(x_a,t)=\W(x_a,t; \vec{x}_b^\beta(t))= U(x_a,t; \vec{x}_b^\beta(t)) + G_a(x_a,t; \vec{x}_b^\beta(t))+i\ J_a(x_a,t; \vec{x}_b^\beta(t))
$$
In a nutshell, we have decomposed the N dimensional \ref{TDSE} into an exact system of N coupled Schrödinger-like Equations for the CWF-s. For each $a \in \qty{1..n}$:
\begin{equation}\label{CWF.SE.III}\tag{CWF.SE.III}
i \hbar \pdv{\psi_a^\beta(x_a,t)}{t} = \qty[\frac{\hbar^2}{2m_a} \pdv[2]{}{x_a} +  U(x_a,t; \vec{x}_b^\beta(t)) + G_a(x_a,t; \vec{x}_b^\beta(t))+i\ J_a(x_a,t; \vec{x}_b^\beta(t))] \psi_a^\beta(x_a,t)
\end{equation}
\begin{equation*}
\begin{split}
\dv{x_a(t)}{t}=v_a(x_a,t)=\frac{1}{m_a}\pdv{\z_a(x_a, t)}{x_a}
\end{split}
\end{equation*}




Es la salvacion, y la unica "ventaja" de las CWF kizas. Ze sale una ecuacion lienal, solucionable de forma estable (aunke no preserve la norma, ze hay potencial complejo). Y es un mix de la full SE y la CE+HJE es un intermitx ke solo es concebible en CWF-s.
% TODO: 
- Generalize G,J zuk guzuzen beste Eulerianera 1h
- Ein Basis Set thing eta por fin harek ekuaziñoiek idatzi 3h
- Leidú Adaptive Gridsen kapituloa 
- Idatzi ALE+suggestion de hacerlas lienales haciendo trajs ke siguen la densidad? 3h
- Leidú dynamical equations for partial derivatives eta idatzi euren atala. 3h
- Leidu lo de Jacobianena eta idatzi suggestion para hacer derivadas respecto a las material coordinates. 3h
- Eing marrazkixek. 3h

\subsection*{(III.b.1.2) Adaptive Grid Equations}
% como puedes controlar als trayectorias dada una monitor function, por ejemplo podrias hacer que conservasen la densidad no? OStyras esa sería muy buena! De fet para esas trayectorias saldria una continuity equation exacta! Y las CWF-s tendrian una time evolution unitaria!!!! Pa sacar cual es el velocity field que deberian seguir quizas podrias mirar cuando la Lagrangian frame equation para las trajs te da que es constante la densidad. Y creo que es cuando la derivada en el espacio de la dSdx aka la velocidad (?) es constante a lo largo de la trayectoria...Y claro como es una fk continua tb deberia funcionar ke exista esa trayectoria no? 



\subsection*{(II.c) Basis Set Expansion}

\subsection*{(III.d) Dynamic Equations for Partial Differentials}


\section*{III . No Pilot-Wave: Finite Tangent Universe Mechanics}




 


\begin{thebibliography}{1}
\addcontentsline{toc}{section}{References}

\bibitem{JordiXO}
	Oriols X, Mompart J,{\em Applied Bohmian Mechanics: From Nanoscale Systems to Cosmology} Pan Stanford, Singapore (2012)
	
%\bibitem{XO}
%	Oriols X. 2007 {\em Quantum-trajectory approach to time-dependent transport in mesoscopic systems with electron-electron interactions} Phys. Rev. Lett. 98 066803

\bibitem{Wyatt}
R. E. Wyatt, {\em Quantum Dynamics with Trajectories} (Springer, Berlin, 2006)

%\bibitem{Dev}
%	Devashish Pandey, Xavier Oriols, and Guillermo Albareda. {\em Effective 1D Time-Dependent Schrödinger Equations for 3D Geometrically Correlated Systems.} Materials 13.13 (2020): 3033.

%\bibitem{nireTFGie}
%	Oyanguren Xabier, {\em The Quantum Many Body Problem}, Bachelor's Thesis (2020) for the Nanoscience and Nanotechnology Degree (UAB).

%\href{https://github.com/Oiangu9/The\_Quantum\_Many\_Body\_Problem\_-Bachellors\_Thesis-/blob/master/TheQuantumManyBodyProblem\_\_BachelorsThesis\_XabierOyangurenAsua.pdf}{https://github.com/Oiangu9/The\_Quantum\_Many\_Body\_Problem\_-Bachellors\_Thesis-/blob/master/TheQuantumManyBodyProblem\_\_BachelorsThesis\_XabierOyangurenAsua.pdf}

%\bibitem{Albareda}
%	Albareda G, Kelly A, Rubio A. {\em Nonadiabatic quantum dynamics without potential energy surfaces.} Phys Rev Materials. 2019; 3: 023803. 

%\bibitem{DATA}
%	All the animations employed for the analysis of Section 3.2 can be found in the following link:\\
%	\href{https://drive.google.com/drive/folders/1vnNDZrIYDlAhd-kVmmnVJgXmcdE2gxAV?usp=sharing}{https://drive.google.com/drive/folders/1vnNDZrIYDlAhd-kVmmnVJgXmcdE2gxAV?usp=sharing}
	
\end{thebibliography}


\end{document}
