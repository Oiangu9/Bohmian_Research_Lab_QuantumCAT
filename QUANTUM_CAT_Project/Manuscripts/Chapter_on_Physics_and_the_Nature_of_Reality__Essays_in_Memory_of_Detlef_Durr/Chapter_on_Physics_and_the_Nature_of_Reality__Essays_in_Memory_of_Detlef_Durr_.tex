\documentclass[11pt, a4paper]{article} % , draft
\usepackage[utf8]{inputenc}

\usepackage{enumitem} % customiçe item dots etc
\usepackage{textgreek} % obv
\usepackage{physics} % for easy derivative notation
\usepackage{amsmath}
\usepackage{amsthm} %theorems
\usepackage{amssymb}
\usepackage{mathtools} % for matrices with blocks inside
\usepackage[scr=boondoxo]{mathalfa}
\usepackage{pst-node}%
\usepackage{mathrsfs}
\DeclareMathAlphabet{\mathpzc}{OT1}{pzc}{m}{it}

\newcommand{\mc}{\multicolumn{1}{c}}
\newcommand{\R}{\mathbb{R}} % command for real R
\newcommand{\Holo}{\mathcal{H}}
\newcommand{\M}{\mathcal{M}}
\newcommand{\C}{\mathbb{C}}
\newcommand{\N}{\mathbb{N}}
\newcommand{\z}{\mathpzc{s}}
\newcommand{\p}{\mathpzc{r}}
\newcommand{\s}{\mathbb{S}}
\newcommand{\W}{\mathbb{W}}
\newcommand{\U}{\mathscr{U}}
\newcommand{\Lg}{\mathscr{L}}
\newcommand{\x}{\mathcal{X}}
\newcommand{\B}{\mathfrak{B}}

\usepackage{csquotes}
\MakeOuterQuote{"}
\setlength{\parskip}{0.3 cm}
\setlength{\parindent}{0 cm}


\usepackage{fancyhdr}

%\usepackage{nath} % authomatic parenthesis stuff
%\delimgrowth=1
\usepackage[left=2cm, right=2cm, top=2.1cm, bottom=2.1cm]{geometry} % set custom margins
\usepackage{graphicx} % to insert figures
\usepackage{grffile}
\graphicspath{{Figures/}} % define the figure folder path
\usepackage{subcaption} % for multiple figures at once each with a caption
\usepackage{multirow} %multirow in tables

\usepackage{caption}
\captionsetup[figure]{font=footnotesize} %adjust caption size
\captionsetup[table]{font=footnotesize} %adjust caption size

\usepackage{booktabs} % for pretty tabs in tables
\usepackage{siunitx} % Required for alignment
\captionsetup{labelfont=bf} % bold face captations

\usepackage{hyperref} % makes every reference a hyperlink
\hypersetup{
    colorlinks=true,
    linkcolor=violet,
    filecolor=[rgb]{0.69, 0.19, 0.38},      
    urlcolor=[rgb]{0.0, 0.81, 0.82},
    citecolor=[rgb]{0.69, 0.19, 0.38}
}

\usepackage{epigraph} % for quotations in teh begginig
\setlength\epigraphwidth{8cm}
\setlength\epigraphrule{0pt}
\usepackage{etoolbox}
\makeatletter
\patchcmd{\epigraph}{\@epitext{#1}}{\itshape\@epitext{#1}}{}{}
\renewcommand{\qedsymbol}{o.\textepsilon.\textdelta}

\newtheorem{prop}{Proposition} %so I can use propositions
\newtheorem{cor}{Corollary} %so I can use corollaries
\newtheorem{defi}{Definition} %so I can use corollaries

\makeatother % all this is for the epigraph
\usepackage{tocloft}

\usepackage{imakeidx} % make index

\makeindex[columns=3, title=Alphabetical Index, intoc]


\title{\vspace{-2cm} {\bf Reality and Causality in the Microscopic World:\\ A Discussion from Quantum Transport Theories}
\vspace{-0.4cm}}
\date{\vspace{-11ex}}
\let\clipbox\relax
\usepackage{adjustbox}
\newcolumntype{?}{!{\vrule width 1.5pt}}
\usepackage{abstract}
\setlength{\absleftindent}{0mm}
\setlength{\absrightindent}{0mm}

\usepackage{tcolorbox}
\DeclareRobustCommand{\mybox}[2][gray!10]{%
\begin{tcolorbox}[   %% Adjust the following parameters at will.
        left=0.2cm,
        right=0.2cm,
        top=0.15cm,
        bottom=0.15cm,
        colback=#1,
        colframe=#1,
        width=\dimexpr\textwidth\relax, 
        enlarge left by=0mm,
        boxsep=5pt,
        arc=0pt,outer arc=0pt,
        ]
        #2
\end{tcolorbox}
}

\usepackage{anyfontsize}


\usepackage{listings}
\usepackage{xcolor}
\lstset{language=C++,
                basicstyle=\ttfamily,
                keywordstyle=\color{blue}\ttfamily,
                stringstyle=\color{red}\ttfamily,
                commentstyle=\color{green}\ttfamily,
                morecomment=[l][\color{magenta}]{\#}
    backgroundcolor=\color{black!5}, % set backgroundcolor
    basicstyle=\footnotesize,% basic font setting
}
%\DeclareMathSizes{display size}{text size}{script size}{scriptscript size}.
\DeclareMathSizes{10}{10}{10}{10}
\setlength{\footnotesep}{0.55\baselineskip}
\begin{document}


\pagenumbering{arabic}
\setcounter{page}{1}
\begin{center}
\section*{ Bohmian Mechanics as a Practical Tool \vspace{0.2cm}\vspace{0.1cm}\\ \small Xabier Oianguren\footnote{\em Departament de Física, Universitat Autònoma de Barcelona, 08193 Bellaterra, Barcelona, Spain}, Carlos F. Destefani\footnote{\em Departament d’Enginyeria Electrònica, Universitat Autònoma de Barcelona, 08193 Bellaterra, Barcelona, Spain}, Matteo Villani$^2$, David K. Ferry\footnote{\em School of Electrical, Computer, and Energy Engineering, Arizona State University, Tempe,
AZ 85287 USA} and Xavier Oriols$^2$}

\vspace{-0.5cm}
{\bf \small - Chapter on {\em “Physics and the Nature of Reality: Essays in Memory of Detlef Dürr” - }}\vspace{-0.32cm}
\end{center}

\hspace*{4mm} Questioning whether "there are" electrons inside our mobile phones sounds like an absurd reflection, and yet the standard (also called Copenhagen or orthodox) quantum theory is not able to affirm it \cite{where, consp}. Under this theory, a quantum object has a well-defined property (like the position) only when its wavefunction is an eigenstate of the associated operator. We know that this happens when the property is "strongly measured". But in general, the wavefunction is in a superposition of eigenstates for the operator of that observable, meaning nothing can be said about it: the property becomes "unspeakable" until measured. Consequently, the Copenhagen theory affirms that it is meaningless to talk about, say, the positions of the electrons inside the active regions of nanoscale devices, because while in operation, their position is never (strongly) measured. Thus, there is no chance for an affirmative answer to our initial question. And yet, consciously or not, no engineer or applied physicist can seriously accept there is no electron in an operating nano-device like a transistor \cite{where, consp}. Fortunately, alternatives to the Copenhagen interpretation of quantum mechanics exist, by which electrons have a defined position irrespective of their measurement and the state of superposition of their wavefunction, e.g., the well-known Bohmian interpretation \cite{Bohm,Holland, Durr,JordiXavier}. \vspace{-0.1cm} 

What might be more relevant from a practical point of view however, is that even if one turns a blind eye to these "picky unspeakabilities" of the Copenhagen theory, their implications also limit the employable modeling tool-set, making some scenarios look (unnecesarily) pathological. For example, the explained undefined position of electrons comes into conflict with a well-defined dwell time for the electrons in the active region of a nano-scale transistor, which is an essential parameter to predict the performance of next generation computers. Similar practical issues can be found in the search of measurement operators (like the multi-electron displacement current \cite{equiv, Pel}) in scenarios where their mathematical shape is far from obvious (e.g. in nano-scale devices operating at THz frequencies \cite{Thz}), or when looking for pure-state "unravellings" in non-Markovian open quantum systems. It is interesting though, that such scenarios are unambiguously resolved under the mentioned Bohmian quantum theory, as we will see. Moreover, it turns out that the followers of the Copenhagen theory that come up with mathematical tools that allow the prediction of the phenomenological manifestations of these “pathological” scenarios, accidentally reach for that, naturally Bohmian concepts like position post-selected weak values or the conditional wavefunction \cite{interpretSSE,NMisModal}.\vspace{-0.1cm}

In this chapter, we will take a trip around several hot-spots where Bohmian mechanics and its capacity to describe the microscopic reality, even in the absence of measurements, can be harnessed as computational tools, in order to help in the prediction of phenomenologically accessible information (equally useful for the followers of the Copenhagen theory). 

We can easily arrive at these conclusions through the inherently Bohmian concepts of a conditional wave-function (CWF) and an effective wave-function (EWF), introduced by {\em Dürr et al.} \cite{Absolute}, together with the understanding of the measurement dilemma they illuminate. We will review them in the first section, along with how the density matrix formalism, and any general quantum operation, can be understood in Bohmian terms. All this will lead us in a natural way to the main theses of the chapter. For instance in section two, we will show how a Stochastic Schrödinger Equation (SSE), when used to compute the reduced density matrix of a non-Markovian open quantum system, necessarily seems to employ CWFs. We will see that by dressing these CWFs with an interpretation, the Bohmian theory can prove to be a useful tool in the search for such SSEs. In section three, we will introduce how a Copenhagen observable operator can be derived from numerical properties of the Bohmian trajectories. Since in Bohmian mechanics the trajectory of an "unmeasured" system is well-defined at all times, we will be free to speak about these numerical properties in the absence of observation. This will be useful even if they are given no ontological meaning, because, not only will we be able to simulate them, but we will see that they can be operationally determined in an experiment. Thus, they will be useful to characterize a quantum system irrespective of the followed quantum theory. \vspace{-0.2cm}


\subsection*{1. Introduction: A Suggestive Review}
\vspace{-0.2cm}

Before going into the details, we note that only basic {\bf non-relativistic quantum phenomena} will be discussed in this chapter. The spirit is to show that, for this kind of phenomena and their formulation, the Bohmian theory provides a most convenient narrative.\footnote{Yet, it is of course possible if not certain, that a deeper level in the quantum theory may require other frameworks.}\vspace{-0.3cm}

\subsubsection*{1.1. The Conditional and Effective Wavefunctions}
\vspace{-0.2cm}

Given a quantum system of $N$ degrees of freedom described by the real coordinate vector $\vec{q}=(q_1,..., q_N)\in\R^N$, we can describe its time-evolution with the use of a complex wavefunction $\Psi(\vec{q},t)=\rho^{1/2}(\vec{q},t)e^{i\mathcal{S}(\vec{q},t)/\hbar}$ (encoding the two real fields $\mathcal{S}$ and $\rho$), and an associated Bohmian trajectory $\vec{q}^{\:\xi}(t)\equiv \vec{q}\:(\vec{\xi},t)$, the initial condition of which is given by the label space vector $\vec{\xi}\in\R^N$, such that $\vec{q}^{\:\xi}(t=0)=\vec{\xi}$. This trajectory is guided by the wavefunction through the "guidance law"\vspace{-0.1cm}
\begin{equation}\label{GL}
\dv{q_k^{ (\xi)}(t)}{t} = v_k(\vec{q},t)\Big\rvert_{\vec{q}=\vec{q}^{\:\xi}(t)}:=\frac{1}{m_k} \pdv{\mathcal{S}(\vec{q},t)}{x_k}\Big\rvert_{\vec{q}=\vec{q}^{\:\xi}(t)},
\end{equation}
while the wavefunction itself is guided by the Schrödinger Equation\vspace{-0.1cm} \cite{Bohm,Holland,Durr,JordiXavier}
\begin{equation}\label{SE}
i\hbar\pdv{\Psi(\vec{q},t)}{t}=\Big[ \sum_{k=1}^N \frac{\hbar^2}{2m_k}\pdv[2]{}{q_k}+U(\vec{q})\Big]\Psi(\vec{q},t),\vspace{-0.1cm}
\end{equation}
where $m_k$ is the mass associated with the $k$-th degree of freedom, $v_k$ is the velocity field piloting the $k$-th degree of the Bohmian trajectories and $U$ denotes the potential describing the interaction between the degrees of freedom (since we consider an isolated system we assume no time dependence on $U$). The most general isolated system we could consider is the entire Universe, where $\vec{q}$ would reflect its degrees of freedom or {\bf configuration}. The fact that a single trajectory $\vec{q}^{\:\xi}(t)$ is assigned to the whole Universe shows the deterministic nature of the Bohmian theory (at least, at the ontological level). 

Let us now partition the entire Universe into a subsystem of interest S, of $n<N$ degrees of freedom $\vec{x}=(x_1,...,x_n)$, and its environment, of degrees of freedom $\vec{y}=(y_{n+1},...,y_N)$; with $\vec{q}\equiv (\vec{x},\vec{y})$. We could associate one wavefunction to the system and one to the environment, both labeled by the initial joint configuration $\vec{\xi}$, as $\psi^\xi(\vec{x},t):=\Psi(\vec{x},\vec{y}^{\:\xi}(t),t)$ and $\varphi^\xi(\vec{y},t):=\Psi(\vec{x}^{\:\xi}(t),\vec{y},t)$. These are particular cases of the so called {\bf conditional wavefunctions}. In general, a CWF is a "slice" of a wavefunction, obtained by evaluating some of its degrees of freedom along a (Bohmian) trajectory, while leaving the rest of them un-evaluated \cite{Absolute, JordiXavier}. 

As proved in \cite{GJ}, the full Schrödinger Equation \eqref{SE}, ruling the dynamics of the entire, can be rewritten exactly into two coupled dynamical sets of equations ruling the motion of the two presented CWFs. Assuming we can write $U(\vec{x},\vec{y}\,)=U_x(\vec{x}\,)+U_{xy}(\vec{x},\vec{y}\,)$, for the system we have\footnote{For the environment they will be the same but changing the CWF and the index ranges.}\vspace{-0.2cm}
\begin{equation}\label{SE.GJ}
i \hbar \pdv{\psi^\xi (\vec{x},t)}{t} = \qty[\sum_{k=1}^n\frac{\hbar^2}{2m_k} \pdv[2]{}{x_k} + U_x(\vec{x}\,)+ U_{xy}(\vec{x}, \vec{y}^{\: \xi}(t)) + G(\vec{x}, \vec{y}^{\: \xi}(t),t)+i\ J(\vec{x}, \vec{y}^{\: \xi}(t),t)] \psi^\xi (\vec{x},t),
\end{equation}
with $G$ and $J$ the real and complex parts of the so-called {\bf quantum correlation potential}
\begin{equation}\label{G.Bohm}\hspace{-0.1cm}
G(\vec{x},\vec{y}^{\:\xi}(t),t):=\sum_{j=n+1}^N\qty[-\frac{1}{2}m_j\qty(v_j(\vec{x},\vec{y}^{\: \xi}(t),t))^2-\frac{\hbar^2}{2m_j\rho^{1/2}(\vec{x},\vec{y}^{\:\xi}(t),t)}\qty(\pdv[2]{\rho^{1/2}(\vec{x},\vec{y},t)}{y_j})\Big\rvert_{\vec{y}=\vec{y}^{\:\xi}(t)} ],
\end{equation}
\begin{equation}\label{J.Bohm}
J(\vec{x},\vec{y}^{\:\xi}(t),t):=-\frac{\hbar}{2}\sum_{j=n+1}^N\pdv{}{y_j}v_j(\vec{x},\vec{y},t)\Big\rvert_{\vec{y}=\vec{y}^{\, \xi}(t)},
\end{equation}
where we recognize as $G$ the difference between the quantum potential \cite{JordiXavier, Durr} and the kinetic energies of the trajectory of the environment; and as $J$, the spatial variation in the environment axes $y_j$ of their associated Bohmian velocity. The evaluation of both $G$ and $J$ involves, at each $\vec{x}$, a derivative of the phase $\mathcal{S}$ or of the magnitude $\rho$ of the full wavefunction $\Psi$ along the environment coordinates $\vec{y}$, centered at the trajectory position $\vec{y}^{\:\xi}(t)$. This means $G,J$ require information about the wave function (about CWFs) over nearby trajectories $\vec{y}^{\:\xi'}(t)=\vec{y}^{\:\xi}(t)+\Delta \vec y$, with $|\Delta\vec y\, |\rightarrow 0$. The peculiarity of having the dynamics of a single CWF linked to the dynamics of other CWFs is known as "quantum wholeness" \cite{JordiXavier}, and it lays in the heart of the discussion about Markovian behavior, as we will see.

Now, we might ask when the subsystem CWF $\psi^\xi(\vec{x},t)$ behaves as if it was an independent closed quantum system wavefunction, ruled by a unitary Schrödinger Equation \eqref{SE}. This happens only while both $G$ and $J$ vanish and $U_{xy}(\vec{x},\vec{y}^{\,\xi}(t))\simeq V(\vec{x},t)$ with a same shape irrespective of the trajectory $\vec{\xi}$.\footnote{ If only $J$ vanished, the CWF would already seem to be ruled by a unitary Schrödinger Equation of a closed system, with a real potential defined as $V(\vec{x},t):=U(\vec{x},\vec{y}^{\:\xi}(t))+G(\vec{x},\vec{y}^{\:\xi}(t),t)$. Computationally though, in order to evaluate $G$ and the trajectory $\vec{y}^\xi(t)$, a qunatum description of the environment would still be required, making the CWF of S not independent of the environment's evolution and thus, not an EWF. } Whenever this is the case, we can say that the CWF of the system is its {\bf effective wavefunction}. The question is then: when do these three conditions happen? One of the most important cases is just after a strong measurement of the subsystem. 

\vspace{-0.2cm}

\subsubsection*{1.2. The Copenhagen collapse law in the Bohmian theory}
\vspace{-0.1cm}

We are routinely interested in subsystems much smaller than the entire Universe. In many circumstances, such subsystems can reasonably be assumed to be modelable effectively as closed systems described by an EWF $\psi(\vec{x},t)$. It is also reasonable to assume that similar subsystems can be prepared in other places and times, each of them described by the same wavefunction $\psi(\vec{x},t)$. We cannot predict with certainty the trajectory $\vec{x}^{\:\xi}(t)$ of each copy (unless we measured it, and then perturb its wavefunction, making them be no longer copies), but by the Quantum Equilibrium principle \cite{Absolute}, if the trajectory of the whole Universe had a "typical" initial condition $\vec{\xi}$, the probability density of the position $\vec{x}$ at time $t$, for any of the experiments, will be given by $|\psi(\vec{x},t)|^2$. The point is that to measure a property of these subsystems, we need to open them temporarily to allow their interaction with a measurement apparatus that will convey the information to us. The Copenhagen theory explains the effect of this interaction by postulating a non-unitary "collapse law". The Bohmian theory can explain its effect using only the unitary evolution \eqref{SE}, as we demonstrate now.

Given the closed quantum system S with EWF $\psi(\vec{x},t)$, in order to take a strong measurement of the property $B$, as suggested by von Neumann \cite{vonNeumann} and explained in terms of Bohmian mechanics in Refs. \cite{Durr, JordiXavier, Holland}, we have to consider the degrees of freedom of a macroscopic measuring apparatus, $z\equiv y_{n+1}$, called "the pointer", as part of the environment of S. It will be a good apparatus if its pointer has a position at the initial time $t=0$, $z^{\;\xi}(0)$, close to its rest position, independently of the rest of the environment. This means it should have an EWF, $\ket{\varphi(0)}_M=\int\varphi(z,0) \ket{z}dz$, with say, $\varphi(z,0)=c e^{-z^2/4\sigma^2}$ and $c:=1/((2\sigma\pi)^{1/4})$, with $\sigma$ small. Then, for the apparatus to extract information of the subsystem S, they are made to interact for a time $T$ through the von Neumann Hamiltonian $\hat{H}_{MS}=\bar{\mu}(t)\,\hat{p}_M\otimes \hat{B}_S$, where $\hat{p}_M$ is the momentum operator of the dial and $\hat{B}_S$ is the operator related with the property $B$ of the system we wish to measure. This is the Hamiltonian that provokes the position of the pointer to get entangled with the possible eigenstates of $B$. Note the measurement strength is defined as $\mu:=\int_0^T\bar{\mu}(t)dt$. Now, if the observable $B$ has countable spectrum, such that $\hat{B}_S=\sum_k b_k \ket{b_k}_S\bra{b_k}_S$, with $\{\ket{b_k}_S\}_k$ an orthonormal basis of S and $\ket{\psi(0)}_S=\sum_k \beta_k(0)\ket{b_k}_S$ with $\beta_k(0)=\bra{b_k}_S\ket{\psi(0)}_S$, the Schrödinger equation \eqref{SE} will cause a unitary evolution $\hat{U} \null_{0}^T=e^{-i\mu T \hat{H}_{MS}/\hbar}$ of the composed initial state $\ket{\Psi(0)}_{MS}=\ket{\varphi(0)}_M\otimes\ket{\psi(0)}_S$ as\vspace{-0.15cm}
\begin{equation}\label{postM1}
\ket{\Psi(0)}_{MS} \overset{\hat{U}_{0}^T}{\longrightarrow} \ket{\Psi(T)}_{MS}=\sum_k \beta_k(0)\qty(\int c \; e^{-\frac{(z-b_k\mu T)^2}{4\sigma^2}}\ket{z}_M dz )\otimes \ket{b_k}_S. \vspace{-0.15cm}
\end{equation}
This means that the probability density for the Bohmian position of the dial $z^\xi(T)$ is concentrated in $z$, in several Gaussians of weights $|\beta_k(0)|^2$. Each $k$-th Gaussian is centered on a different $z=b_k \mu T$. The interaction strength $\mu$ must be big for the dial to determine a single result. That is because in this way the Gaussians can have a disjoint support, meaning that around each Gaussian the CWF for the system S is always proportional to one of the eigenstates $\ket{b_k}_S$. This eigenstate will be the one with eigenvalue $b_k$ with $b_k\mu T$ closest to the Bohmian position of the dial $z^\xi(T)$. The CWFs linked to the rest of Bohmian positions of the dial $z^{\xi'}(T)$, are the so-called "empty waves". At this point, the Copenhagen theory invokes a random and spontaneous physical phenomenon, called the wavefunction "collapse", that transforms the wavefunction \eqref{postM1} into a product of a single eigenstate $\ket{b_k}_S$ and its corresponding Gaussian, with a probability $|\beta_k|^2$ \cite{vonNeumann}. The observation introduced by the Bohmian theory is that the Gaussians modulating different CWFs (different $\ket{b_k}$) must be {\bf macroscopically disjoint} in $z$, since only in this way can the dial macroscopically show the different measurement results. This makes the correlation potentials $G$ and $J$ of Eqs. \eqref{G.Bohm}, \eqref{J.Bohm}, between the dial M and the subsystem S vanish. Importantly, since any interaction between M and S vanishes for significant times $t>T$, the potential energy will be separable as $U(\vec{x},z)=U_x(\vec{x})+U_z(z)$. As a consequence, the macroscopic separation of the other CWFs (the "empty waves") will only get more disjoint \cite{Absolute}, making $G$ and $J$ zero at any significant future time. This means, the CWF for S can be evolved for $t>T$, as if it were again an independent closed quantum system wavefunction: it is an EWF. From the perspective of S, omitting $t\in(0,T)$, it will have looked like the state $\ket{\psi}$ suffered a "collapse" into the $\ket{b}_k$ indicated by the dial $z^\xi(T)$. By the Quantum Equilibrium hypothesis \cite{Absolute}, the $k$-th state will be indicated with the weight modulating its Gaussian envelope, $|\beta_k(0)|^2$, naturally implying a phenomenological accordance with the Copenhagen collapse law.


Notice that, either the assumption that for time $t>T$, M does not interact anymore with S, or that the environmental entanglement with S is lost by some sort of thermalisation (by which the empty waves $-$ the rest of CWFs that are not "sliced" at $z=z^{\:(\xi)}(T)$ $-$ get macroscopically dispersed), means that the information of the subsystem "leaked" to the environment M, the "empty waves" themselves, do not interact back with the EWF of S. Any of these two assumptions thus imply that the environment effectively "forgets" the entanglement achieved with S. This is an environment behavior we could call memory-less or {\bf Markovian}. Then, because the description of the environment M will only be "useful" for the measurement time interval $(0,T)$, and can then be discarded, we can instead explain the apparent projection of the state of the subsystem into an eigenstate of the operator B, $\ket{\psi(t)}\to \ket{b_k}_S$, with a set of effective-"collapse" orthogonal projectors $\{\hat{\Pi}_k\}_k$ (with $\hat{\Pi}_k=\ket{b_k}_S\bra{b_k}_S$), without the need to explicitly formalize M and the whole interaction process \cite{Durr}. A Bohmian description shall not forget however, that this is just a shortcut in the modeling.\vspace{-0.3cm}



\subsubsection*{1.3. How to use the Effective Collapse for a more general Measurement}\vspace{-0.1cm}

Using the effective collapse idea described above, we can extract more general information about the subsystem. If part of the environment gets entangled with S and this environment portion then suffers an effective collapse as in the strong measurement, S will also seem to suffer an effective "collapse", but now into non-necessarily orthogonal, nor linearly-independent, states (nor a number of states limited by the dimensionality of the Hilbert space of S). It will seem to suffer a non-unitary, non-orthogonal projection. If such an effective collapse of the environment was caused by a strong measurement by an observer, this is called a {\bf generalized measurement} \cite{Generalized, Durr}. Let us explain it in more detail.

Given the initial state $\ket{\psi}_S$ of S, and a fiducial state $\ket{\theta}_A$ for an ancilla A, using an entangling unitary $\hat{U}_{AS}$, we get: $\hat{U}_{AS}\ket{\theta}_A\otimes\ket{\psi}_S=\sum_m \ket{m}_A\otimes \ket{\psi_m}_S$, where $\ket{\psi_m}_S:=\bra{m}_A\otimes\hat{I}_S \big(\hat{U}_{AS}\ket{\theta}_A\otimes\ket{\psi}_S\big)$ is an unnormalized S state called the $m$-th {\bf conditional state} of S, which is entangled with the state $\ket{m}_A$ of A. Note that the conditional state $\ket{\psi_m}_S$ is in fact a subsystem CWF, obtained by evaluating the dial position of M indicating the state $\ket{m}_A$ in the composite wavefunction. If we now perform a strong measurement on A for the orthonormal $\{\ket{m}_A\}_m$ basis (coupling a dial M to A and branching A in EWFs consisting of the measured eigenstates), an ancilla-subsystem EFW proportional to $\ket{m}_A\otimes \ket{\psi_m}_S$ would be generated as a function of the Bohmian position of the dial M. By the Quantum Equilibrium principle \cite{Absolute}, the $m$-th result would be observed with a probability equal to the norm squared of its conditional state $P_m:=|\bra{\psi_m}_S\ket{\psi_m}_S|^2$. If we then turn off the interaction between A and S for all future times, the subsystem S will have seemed to "collapse" into the EWF $\ket{\psi_m}_S/\sqrt{P_m}$ with probability $P_m$, since it will now evolve independently of $A$ (again, we treat $A$ as an environment for the subsystem S with a Markovian behavior). 

Consequently, as with the projective measurement of S, we can shortcut the formalization of A and its measurement, by just considering a set of general measurement (linear) operators on S (called positive operator-valued measures or POVMs) $\big\{\hat{\Omega}_m:=\bra{m}\otimes\hat{I}_S\:\hat{U}_{AS}\ket{\theta}_A\otimes\big\}_m$, sending the S states to post-measurement (unnormalized) EWFs as: $\hat{\Omega}_m\ket{\psi}_S$. The only requirement for them is: $\sum_m \hat{\Omega}_m^\dagger\hat{\Omega}_m=\hat{I}_S$, so that the $m$-probabilities add-up to unity. This is satisfied because $\hat{\Omega}_m\ket{\psi}_S=\ket{\psi_m}_S$ is the (unnormalized) post-measurement EWF of S, the squared norm of which is the probability $P_m$ to observe $m$ and thus get the EWF $\ket{\phi_m}_S:=\ket{\psi_m}_S/\sqrt{P_m}$ for S \cite{Generalized, Durr}. The reason why such an A and $\hat{U}_{AS}$  exist for {\bf any} set of linear operators $\{\hat{\Omega}_m\}_m$ satisfying the stated restriction, will be seen in a moment. Note that, the orthogonal projectors $\hat{\Pi}_k$ of the strong measurement are naturally a subclass of generalized measurement POVMs.\vspace{-0.2cm}

\subsubsection*{1.4. The Density Matrix in this Bohmian Narrative}\vspace{-0.15cm}
Let us now motivate the \textbf{density matrix} formalism from our Bohmian perspective. Consider the state $\ket{\psi(0)}$ and the measurement POVMs $\{\hat{\Omega}_m\}_m$ for a measurement lasting a time $T$, that leaves $\ket{\psi(0)}$ in the possible EWFs $\ket{\phi_m(T)}:=\hat{\Omega}_m\ket{\psi(0)}/\sqrt{P_m}$, with associated probabilities $P_m:=\bra{\psi(0)}\hat{\Omega}_m^\dagger \hat{\Omega}_m\ket{\psi(0)}$. If we were not interested in a particular outcome, but we wanted to keep track of all possible post-measurement EWFs from time $T$ to $t>T$, that is, we would like to study the {\bf unconditional}\footnote{Meaning we keep track of all possible outcomes of the measurement.} post-measurement system, we could first organize them in a set of vector-probability pairs $\Lambda_{T}:=\{\big(\ket{\phi_m(T)},\ P_m\big)\}_m$, and unitarily evolve its wave-vectors with say, $\hat{U}_{T}^{t}$, such that $\Lambda_{t}:=\big\{\big(\hat{U}^t_T\ket{\phi_m(T)},\ P_m\big)\}_m=\{\big(\ket{\phi_m(t)},\ P_m\big)\}_m$. Now, if a second measurement took place, lasting a time $T'$, with effective projection operators $\{\hat{\Omega}'_j\}_j$, we would need to branch each of the independent states in $\Lambda_t$ into the corresponding projected states and update their probabilities: $\Lambda_{t+T'}:=\big\{ \big(\hat{\Omega}_j'\ket{\phi_m(t)}/\sqrt{P_{(j|m)}},\ P_mP_{(j|m)}\big)\big\}_{m,j}$ with $P_{(j|m)}:=\bra{\phi_m(t)}\hat{\Omega}^{'\dagger}_j\hat{\Omega}_j\ket{\phi_m(t)}$ the probability to get the $j$-th outcome, if we obtained the $m$-th one previously. Importantly, the operator $\hat{\Omega}'_j$ could happen to send any state to the same space: for example $\hat{\Omega}'_j=\ket{\varphi_j}\bra{\varphi_j}$. In consequence, the normalized state $\hat{\Omega}'_j \ket{\phi_m}/\sqrt{P_{(j|m)}}$ would be the same state $\ket{\varphi_j}$ for all $m$, and we would need to unify them in our $\Lambda_{t+T'}$ set, updating the probability for this state with the contributions of all the cases that lead to it: $P_{(j|\cdot)}:=\sum_m P_mP_{(j|m)}$. But this is quite cumbersome. It is true that to have a full Bohmian description, we should keep track of the possible Bohmian trajectories for each wave-vector. However, if we could afford the loss of microscopic Bohmian detail (as done in the Copenhagen approach) and just cared for phenomenological probabilistic predictions like $P_m$ or $P_{(j|\cdot)}$, there would be a convenient structure that works "automatically" as our sets of states $\Lambda_t$. This is a "matrix" operating on wave-vectors in which we set the possible EWFs as its composing "ket-bras", with the probability of each EWF as coefficient. This positive-definite matrix used as "possible state-probability" pair container is called a {\bf "density matrix"} \cite{vonNeumann, Durr, Holland}. 

It is trivial to see that such a matrix works as we wanted for post-measurement states. For example, in order to get the post-measurement density matrix for the first measurement, $\hat{\rho}(T) = \sum_m P_m\ket{\phi_m(T)}\bra{\phi_m(T)}$, we can just take the matrix for the initial state $\hat{\rho}(0):=\ket{\psi(0)}\bra{\psi(0)}$, and compute $\hat{\rho}(T)=\sum_m \hat{\Omega}_m^\dagger\hat{\rho}(0)\hat{\Omega}_m$. Then, if we apply a unitary evolution $\hat{U_T^t}$ (or in fact any linear operator) at each side of the matrix (the Hermitian conjugate in the left), we will be applying the operator to each state-vector inside it, "independently of the rest": $\hat{\rho}(t)=\hat{U}_T^{t\,\dagger}\hat{\rho}(T)\hat{U}_T^{t}=\sum_m P_m\ket{\phi_m(t)}\bra{\phi_m(t)}$. Then, for the second measurement we repeat the trick and now the same states $\ket{\varphi_j}$ will automatically be gathered in a single "ket-bra". Not only the time evolution and successive effective-collapses are easily managed with such a structure, but single-time measurement probabilities for an arbitrary operator $\hat{\Omega}_\alpha$ are easily computed as: $P(\alpha)=tr[\hat{\Omega}_\alpha^\dagger \hat{\Omega}_\alpha\hat{\rho}]$, where $tr$ is the trace operation.

Although very useful for ensemble single-time statistical predictions in effective collapse scenarios, for a proper Bohmian approach, one should always remember that the microscopic details of what is happening are lost when using these density-matrices, both regarding the Bohmian trajectory and the state-vectors themselves (since different vector-probability sets yield the same matrix). Importantly, the state-vector and the trajectory of the Universe are still the essential state descriptors in the Bohmian theory, because the Universe as a whole should be described by a determined state-vector.

The so called "partial trace" operation is tightly related to these post-measurement density-matrices. If the density matrix of an ancilla-subsystem composite is $\hat{\rho}_{AS}$, given an arbitrary orthonormal base of A, $\{\ket{m}_A\}_m$, the partial trace of $\hat{\rho}_{AS}$ over A is defined as: $tr_{A}[\hat{\rho}_{AS}] := \sum_m \bra{m}\otimes \hat{I}_S(\hat{\rho}_{AS})\ket{m}\otimes \hat{I}_S$ \cite{Generalized, Durr}. We call this result, the {\bf reduced density matrix} of S. Its relation with measurements comes from the following observations. It turns out that the partial trace of M on the composite pure state $\hat{\rho}_{MS}(T)=\ket{\Psi(T)}_{MS}\bra{\Psi(T)}_{MS}$ describing the post-measurement state of Eq. \eqref{postM1} (under the effective "collapse" conditions for $\mu,T,\sigma$), yields the unconditional post-measurement density matrix of S, made by the possible "collapsed" EWFs $\rho(T)_S=\sum_k |\beta_k|^2\ket{b_k}_S\bra{b_k}_S$. The same happens in any generalized measurement: the partial trace of A in the state $\hat{U}_{AS}\ket{\theta}_A\otimes\ket{\psi}_S$ following the notation of the previous section, will yield the unconditional post-measurement density matrix $\hat{\rho}_S=\sum_m P_m \ket{\phi_m}_S\bra{\phi_m}_S$. In general, this indicates that the partial trace of an ancilla partition A of a composite AS space, can always be interpreted as {\bf how the subsystem S would be left if an unconditional strong measurement was performed on A }\cite{Generalized}\footnote{In the literature \cite{density} it is usual to define the "statistical density matrix" as the one where the uncertainty is due to the lack of human knowledge, meaning its states evolve unitarily. Meanwhile in the "reduced density matrix" the uncertainty has its origin in the fact that we only look at the subsystem. In general, its states are conditional states that will not have a unitary evolution. Yet, in some cases a density matrix can be of both kinds simultaneously. In this chapter we exclusively talk about "reduced density-matrices" and include the "statistical matrices" as a sub-case in which there has been an effective collapse with random outcomes, leaving those in which the uncertainty is not of "quantum origin" (the statistical matrices that are not reduced ones) out.}. Importantly, if the traced partition is not projectively measured (coupling a measurement pointer to it and evolving until macroscopic distinguish-ability is achieved) and the interaction between A and S is not "thermalised" and does not decay indeterminately, then the reduced density matrix of S will just be a "fiction". Each conditional state of S, each $\ket{\phi_m}_S$, (which we placed in different slots of the matrix after the partial trace) will still interact with each other since they are not EWFs. Therefore, even if the reduced density matrix is enough to predict measurement statistics on S at the time of the partial trace, in typical (non-Markovian) scenarios it will not convey enough information to predict its time evolution. The Bohmian information of the distribution of the CWFs slicing the composite wavefunction would be required for that (which is encapsulated in the standard quantum theory in the so called "memory	time superoperator" \cite{WisemanSSE} of non-Markovian master equations). To know this without explicit mention of the environment is not trivial in either the Copenhagen or Bohmian approaches.\vspace{-0.25cm}

\subsubsection*{1.5. General Quantum Operations within the Bohmian Narrative}\vspace{-0.2cm}

In order to finish integrating the density matrix formalism and any general quantum operation (including the generalized measurements) with this Bohmian view, we can invoke the Gelfand-Naimark-Segal theorem \cite{GNSTheorem, Generalized}. According to this theorem, we can assure that, for any general operation we can perform on a density matrix $\hat{\rho}_S$ of a system S (any complete-positive, convex linear and not trace increasing superoperator acting on S), let us call it the operation $\mathfrak{S}$, there exists at least an ancilla system A with a pure state $\ket{\theta}_A$ and a coupling unitary evolution $\hat{U}_{AS}$ such that\vspace{-0.15cm}
\begin{equation}
\mathfrak{S}[\hat{\rho}_S]=tr_A\qty[ (\hat{\Pi}_A\otimes \hat{Id}_S)  \hat{U}_{AS}\qty(\ket{\theta}_A\bra{\theta}_A\otimes \hat{\rho}_S)\hat{U}_{AS}^\dagger],
\end{equation}
which can be interpreted as a unitary coupling of the initially independent S and A, and posterior partially unconditional strong measurement of $A$ (where only the eigenstates of non-null eigenvalue of $\hat{\Pi}_A$ are left and the rest are discarded). We have given a Bohmian view for all of them, so this closes the entire basic (non-relativistic) picture. Interestingly, this is a practical method used by quantum engineers to physically implement arbitrary quantum operations $\mathfrak{S}$ in a lab.\vspace{-0.2cm}


\subsection*{2. How non-Markovian SSEs employ CWFs}
Consider an evolution of the reduced density matrix of a subsystem S that could be equivalently interpreted as if a different portion of the environment (a different ancilla) instantly got coupled every $\Delta t$ with S and was then ideally measured. If these ancillas never again interacted with the system (or their entanglement was somehow "thermalised" before their next interaction), the result would be equivalent to a generalized measurement of S every $\Delta t$. Then, following our previous comments, we could call such a system S, a Markovian open quantum system \cite{QuantumTrajs}. Among others, the "Past-Future Independence" definition of Markovian behavior in Ref. \cite{MarkovianityDefs}, perfectly matches this view.

In fact, as shown by Ref. \cite{continousMeas}, such a continuous monitorization of different ancillas, that get coupled to the subsystem at each small time-step, can be used to derive dynamical equations (master equations) for the reduced density matrix in some Markovian environments. Then, the derivation of a Lindblad master equation \cite{Generalized, MarkovianityDefs} for an arbitrary Markovian environment just requires the consideration of several simultaneous continuous measurements for different properties of the bath \cite{continousMeas, MarkovianityDefs}.

Because the dynamics of the reduced density matrix of a subsystem S can be understood in these terms, instead of trying to solve a Markovian master equation directly, we could do the following. First, find some (fictitious or not) environmental ancillas E and an observable $W$, such that if the ancillas get entangled with S one after the other, and their property $W$ is sequentially (projectively) measured, they cause the same (unconditional) evolution of the reduced density matrix of S, as the one described by the master equation of S. As mentioned, this is always possible for a Markovian open quantum system S. Then, we could evolve a pure state-vector of S, by choosing for each projective measurement of the bath ancillas, one of the possible post-measurement conditional states. This would generate pure state $\ket{\psi_{w(t)}(t)}_S$, associated with the result of a certain chain of measurements (an unravelling) of the bath ancillas: $w(t)$.\footnote{At each time a different generalized measurement is performed on S, meaning the stochastic trajectory $w(t)$ reflects the Bohmian positions of different measurement dials at each $\Delta t$. Its non-differentiability is thus unproblematic.} This pure state is called a {\bf quantum trajectory}, linked to a "noise realization" $w(t)$ for its environment \cite{Generalized, MarkovianityDefs, QuantumTrajs}. As we saw previously, the reduced density matrix defines what would be left if an unconditional ideal measurement was performed on the environment of the system. This tells us that we can finally recover the reduced density matrix for S by averaging the ensemble of all possible quantum trajectories for the unravelling of the observable $W$ in the bath ancillas \cite{MarkovianityDefs,QuantumTrajs}\vspace{-0.17cm}
\begin{equation}
\hat{\rho}_S(t):=tr_{E}[\hat{\rho}_{ES}(t)]=\mathbb{E}_{w(t)}\qty[\ket{\psi_{w(t)}(t)}_S\bra{\psi_{w(t)}(t)}_S]. \vspace{-0.07cm}
\end{equation}
Computationally, this means that if for a given master equation, we got the stochastic equation ruling the time evolution of such quantum trajectories $\ket{\psi_{w(t)}}_S$, we would be able to parallelize the computation of the reduced density matrix by solving vector equations, instead of a matrix equation \cite{MarkovianityDefs, QuantumTrajs}. As an additional advantage, the reduced density matrix obtained this way is positive semi-definite by construction. Equations of this kind are the so-called, {\bf Stochastic Schrödinger Equations} \cite{Generalized, continousMeas}. Note that such a pure state trajectory for a Markovian environment E, can always be physically interpreted in the Copenhagen explanation, as a so-called pure unravelling \cite{MarkovianityDefs} (where one would invoke the collapse of the subsystem wavefunction at each $\Delta t$). In the Bohmian view, a quantum trajectory is just a normalized CWF of the subsystem S (in ket notation) which is converted into an EWF (thus the normalization), after every significant $\Delta t$.\vspace{-0.05cm}
 

However, what if we had an environment E that gets entangled with S, but which never really allows us to consider an effective collapse? What if the different CWFs of the subsystem S were allowed to interact in any future time, and were not converted into EFW every $\Delta t$? That is, what if the "quantum trajectories" could interact between themselves, such that the evolution of each of them depended on the rest? Then "the information leaked" into the environment from S (the "empty waves"), would be able to affect back the evolution of S at any significant future time. Such an environment with "memory" of the entanglement achieved with S could be called a non-Markovian environment \cite{MarkovianityDefs}. It turns out that, from a Bohmian interpretation, we could still continue talking about "pure state quantum trajectories", which would be the CWFs for the subsytem S (in any desired representation), conditioned on a position for the environment interacting with S (or conditioned on the positions of dials coupled with arbitrary observables of the environment ancillas interacting with S $-$ to allow SSEs in non-position representations) \cite{NMisModal, interpretSSE}.
\vspace{-0.05cm}

Contrarily, in the Copenhagen view, a CWF (whether normalized or not and in any representation basis), does not have a physical interpretation, unless it is an EWF, e.g. unless the conditioning variable is projectively measured. As a consequence, if a SSE is found for a non-Markovian dynamical equation ruling a reduced density matrix, the conditional pure state evolved by the SSE, in the Copenhagen view, can only be understood as the state in which S would be left in, if the environment E was measured. But, since it is non-Markovian, we cannot assume E is being projectively measured. If it was, the evolution of the state would be substantially different (we would neglect part of the quantum wholeness, the interaction between the CWFs stacked along the coordinates of E). Thus, the linking of such states in time, can only be understood if we leave the Copenhagen picture and use concepts like the CWF of the Bohmian view \cite{NMisModal, interpretSSE}. Of course, mathematically, one could derive such non-Markovian SSEs as pragmatic computational tools to reconstruct the reduced density matrix, but one would need to avoid any additional consideration for the quantum trajectory (like two-time correlation computations), unless one accepts some sort of ontological reality (independent of measurement) for the conditioning property of the environment. But again, this then suggests the usage of the Bohmian (or some other modal) theory.

From the Bohmian perspective, it is easy to realize why SSEs for non-Markovian environments will rarely be exact. One of the main properties a SSE needs to have is that it should allow the time evolution of a single conditional state independently of the rest of possible conditional states (allowing parallelization). This would forbid any dynamical influence between adjacent CWFs. Such an influence however, is the main signature of quantum mechanics (the quantum wholeness). As we saw, if it vanished, this would make the CWFs be EWFs, which would consequently allow a Markovian interpretation of the SSE, implying a contradiction. Yet, SSEs that {\bf approximate} the dynamics for ad-hoc cases are indeed possible in non-Markovian environments \cite{ Diosi, WisemanSSE, Thz}. This is because an ensemble of CWFs does not need to be an ensemble of EWFs to allow the computation of the reduced density matrix at each time.\footnote{In fact, a whole set of CWFs, if the system state was not mixed, would also allow the reconstruction of the full environment-system wavefunction. In which case a reduced system density matrix would not even be necessary.} To see that this is so, independently of the nature of the environment, let us prove it for an arbitrary composed pure state (the generalization to mixed states is then trivial). Given the arbitrary state $\ket{\Psi}_{AS}$ for the environment A and system S, with position observables $\vec{y}$ and $\vec{x}$ respectively, just as introduced in the beginning\footnote{Note that since Bohmian trajectories do not cross each other in configuration space, if we sampled only Bohmian trajectories for which $\vec{x}(0)$ is a single position $\vec{a}$, at each time, we would still have a CWF per each position in $\vec{y}$. Thus, we have that the states $\qty{\ket{\psi^{y(\xi,t)}(t)}_S:=\bra{y(\xi,t)}_A\ket{\Psi(t)}_{AS};\ \vec{x}(0)=\vec{a}}_{y\in\R^{N-n}}$ are all the possible slices of the $\vec{y}$ axis.}\vspace{-0.2cm}
\begin{equation}
\ket{\Psi(t)}_{AS}=\int\ket{\vec{y}^{\:\xi}(t)}_A\otimes \ket{\psi^\xi(t)}_S d\xi = \int\ket{\vec{y}}_A\otimes \ket{\psi^{y(\xi,t)}(t)}_S dy.
\end{equation}
Then tracing out A in $\hat{\rho}_{AS}(t)=\ket{\Psi(t)}_{AS}\bra{\Psi(t)}_{AS}$, would yield the reduced density matrix for S
\begin{equation}\hspace{-0.3cm}
tr_A\qty[\hat{\rho}_{AS}(t)] = \int\bra{\vec{y}}\hat{\rho}_{AS}(t)\ket{\vec{y}}dy = \int \ket{\psi^{y(\xi,t)}(t)}\bra{\psi^{y(\xi,t)}(t)} dy = \mathbb{E}_{y(\xi,t)}\qty[\ket{\psi^{y(\xi,t)}(t)}\bra{\psi^{y(\xi,t)}(t)}],
\end{equation}
which proves the ensemble average of the CWFs reproduces the reduced density matrix at any time.\vspace{-0.15cm}

This narrative in terms of Bohmian CWFs for non-Markovian open quantum systems is not only theoretically insightful, but is a {\bf practical} tool to look for reasonable SSEs. 

\subsubsection*{2.1. Non-Markovian SSE for nano-electronic devices operating at THz frequencies}\vspace{-0.2cm}
In the first section, we arrived at an exact system of equations \eqref{SE.GJ}, that described the general time evolution of CWFs in arbitrary settings. In principle, in those equations the CWF of the subsystem S and its environment E are coupled at all times, not only between them, but also with the rest of possible CWFs (signature of the non-Markovianity). However, for specific scenarios, we can make educated guesses for the quantum correlation terms $G$ and $J$ \eqref{G.Bohm},\eqref{J.Bohm}, and the classical potential $U$, to generate a SSE for individual CWFs of the subsystem S. Thus, equations \eqref{SE.GJ} are a general framework to look for SSEs. In fact, they are also a way to detect non-Markovian behavior. As long as the CWFs of the susbsytem S, $\psi^\xi(\vec{x},t):=\Psi(\vec{x},\vec{y}^{\:\xi}(t),t)$, are described by $G$, $J$ or $U$ that depend explicitly on $\vec{y}^{\:\xi}(t)$ at all times, the system will be notably non-Markovian.

As a practical example, we developed the BITLLES simulator \cite{tdp,Pois,Thz}. In this simulator, we consider a nano-scale electronic device operating at high frequencies (in the order of THz), where both the relevant dynamics of the active region electrons and the current measurement times are in the sub-picosecond ranges. These times imply that the active region of the device cannot be considered to be a Markovian subsystem \cite{Thz}. Within the language of Eq. \eqref{SE.GJ}, the simulator computes the potential $U$ as a solution of the Poisson equation \cite{Pois}, while $G$ and $J$ are modeled by proper boundary conditions \cite{boundary1, Pois} that include the correlations between the active region of the two-terminal nano-device and the reservoirs. Even electron-phonon decoherence effects can be included \cite{eph}.

Following the notation of Eq. \eqref{SE.GJ}, the electrons contributing to the electrical current, the observable of interest, are mainly the $K:=n/3$ electrons inside the active region of such a nano-device. This active region is the subsystem S of interest. The number $K$ fluctuates in time as there are electrons entering and leaving this region, leading to an abrupt change in the degrees of freedom of the CWF of S. This problem can be circumvented  by further decomposing the CWF of S, $\psi^{\xi}(\vec{x},t)$, into a set of CWFs for each electron. That is, for each of the $K$ electrons of position $\vec{x}_m\in\R^3$, we define a single-particle CWF $\phi_m^\xi(\vec{x}_m, t):=\psi^{\xi}(\vec{x}_m, \vec{x}_{\neg m}=\vec{x}_{\neg m}^{\:\xi}(t),t)$ with $\vec{x}_{\neg m}:=(\vec{x}_1,..,\vec{x}_{m-1}, \vec{x}_{m+1}, ...,\vec{x}_{K})$. Then, we can consider a set of $K(t)$ equations like Eq. \eqref{SE.GJ}, one for each of the electrons, to evolve each single-particle CWF $\phi_m^\xi$. What we have just done is to consider S, as itself composed of several open quantum systems that will interact with each other in a non-Markovian way. At the cost of knowing the ad-hoc approximation for their $U,G$ and $J$ potentials explained in the previous paragraph, this turns out to simplify the problem, because we just need to solve $K$ single-particle equations rather than one $K$-particle equation.

The active region of the electron device S, is connected by a macroscopic cable A (represented as an ancillary environment that gets entangled with S) to an ammeter M (acting as a measuring apparatus). The electrical current read by the Bohmian position of the dial in the ammeter M is the phenomenological observable we want to predict. Thus, in principle, the evaluation of the electrical current should require keeping track of all the environmental degrees of freedom of A and M. However, it is known that the total current density (in the time invariant case), defined as the sum of the particle and displacement currents, is a divergenceless vector \cite{diver1, diver2}, which makes the total current evaluated at the ends of the active region S, equal to the total current evaluated at the cables A. It is additionally known \cite{equiv} that since the cables have macroscopic dimensions, the current at the ends of the active region S is only due to the electrons in this region (plus a nearly white noise).\footnote{This is qualitatively because the electrons in the metallic cables have a very short screening time, meaning the electric field generated by an electron in the cable spatially decreases rapidly due to the presence of many other mobile charge carriers in the cable that screen it out. Thus, the contribution of these outer electrons of A to the displacement current at the border of the active region is negligible \cite{neg}.} Therefore, by disregarding the nearly white noise introduced by the cables, the variable of the environment associated to the total current $z(t)\equiv I(t)$ can be equivalently computed at the boundary of S, by only considering the electrons inside S. However, this is only conceivable within Bohmian mechanics. The description of this scenario with the Copenhagen theory faces the problem that there is no measuring apparatus at the boundary of S. Even if we assume a far from S measuring apparatus, it is not at all clear what the eigenstates of the operator measuring the total current will be. Thus, in the Copenhagen language, it is a big challenge to even imagine the system we have described. In the Bohmian description however, not only is it a feasible task as we saw, but its predictions are in accordance with phenomenology (what we end up measuring), as we will detail in the last section.

In Figure \ref{fig:fig} the ability of the present method is demonstrated \cite{inject,Thz} by predicting for a field-effect transistor (FET) with a graphene channel, the time needed to acknowledge a stable reaction of the drain and source currents when the gate voltage is changed. The Klein tunneling suffered by the electrons while traversing the channel (partition noise) and the random energies of the electrons when injected into the system (thermal noise), cause a fluctuation in the instantaneous current that can be diminished by window averaging. The required window size for the change to be consistent for digital applications (binary messages) defines the operating frequency of the transistor.\vspace{-0.2cm}
\begin{figure}[h!]
  \centering
  \hspace*{-0.5cm}
   \includegraphics[width=1\linewidth]{Figures/figurefinal.pdf}\vspace{-0.3cm}
   \caption{(a) Schematic representation of the graphene-based FET, with a channel composed of a single-crystal mono-layer graphene. (b) The high frequency lines are the instantaneous currents (time-averaged at a window of $0.03$ ps) as a function of time. The straight lines are due to a wider averaging window of 4 ps, where we can clearly assert the binary response. We can conclude that, 4 ps is a reasonable operating time for the transistor.}\vspace{-0.4cm}
  \label{fig:fig}
\end{figure}

 
\subsection*{3. Speakable and Operational Information about an Unmeasured system?}

Returning to the discussion at the beginning of the chapter, under the Copenhagen {\em eigenstate-eigenvalue link}, we can only say that a quantum system {\bf has} a defined property when its wavefunction is an eigenstate of the operator related to that property. Since a strong measurement, as we have seen, always forces the system to adopt an eigenstate, while the unitary evolution in the meantime, will typically cause a superposition, it seems we are only allowed to speak about properties of {\bf measured} quantum systems. This makes the predictions about what (strong) measurement apparatus pointers show, privileged in front of the rest of the information computable using the state of the pre-measurement quantum system. It is true that a quantum theory that correctly predicts what the measurement apparatus pointer will show, is by construction enough for phenomenological predictions. This is why it is argued (even by some Bohmian physicists) that if these predictions are obtainable with empirical agreement, dealing with the rest of the information concealed in the system's state (before its interaction with the measurement apparatus), is just adding unnecessary controversy. However, there are scenarios where the characterization of a quantum system, without the inclusion of the "collapse backaction" by a measurement pointer, would solve serious {\bf practical} difficulties.

Paradigmatic example of such scenarios are those cases when two-time information about a closed quantum system is required. For instance, when trying to understand the maximum working frequency of nano-scale transistors to test the performance of modern computers \cite{modern}, the time spent by electrons in the active region of the transistors, their {\bf dwell time}, must be measured. The eigenstate-eigenvalue link would force us to place position detectors in the two ends of the active region. However, the quantum measurement, no matter how weak it is, introduces an effective collapse backaction in the system that disrupts its future evolution. Thus, the number given by these detectors would be meaningless to benchmark "unmeasured" transistors: no current computer has position detectors at the ends of its transistors \cite{tunnel1, tunnel2}. An additional example happens in thermodynamics: because {\bf work} is by definition a dynamical property implying knowlege of the system (at least) at two different times, it seems there is no possible universal definition for a quantum work operator \cite{nogo, workPb1, workPb2}. Perhaps more generally, {\bf two-time correlations} of non-commuting observables, say $F$ and $B$, cannot be defined without including an explicit disturbance by a particular measurement scheme. We could correlate the result of a strong measurement of $F$ at time $t_1$ and a strong measurement of $B$ at time $t_2$, but since the measurement at $t_1$ will project the state to different EWFs, the disturbing backaction of the measuring device will be obvious.\footnote{ Even numerically, it seems hard to describe well-defined correlations. If their operators do not commute, the expectation $\langle \hat{B}(t_2)\hat{F}(t_1)\rangle$, in the Heisenberg formalism, will typically give a complex number. We could just take the real part $\mathbb{R}e \big\{\langle \hat{B}(t_2)\hat{F}(t_1)\rangle\big\}$, which turns out to be the correlation of a weak measurement \cite{Weak} of $F$ at time $t_1$ and a strong measurement of $B$ at time $t_2$. Yet, as shown in \cite{DevInPosition2}, even an ideal weak measurement does in fact perturb the system in a way that is dependent, for example, on the particular fiducial state employed for it.} Thus, are we fundamentally forbidden to access dynamical information about the "unmeasured system"\footnote{A system that is not being measured, e.g. a closed system evolving without quantum interaction with its environment.}? Or is there a way to consistently define non-{\em contextual}\footnote{Contextual means it depends and implies the environment used to convey the information to the observer.} quantum properties? Bohmian mechanics, with its ontology of reality being persistent even when no measurement is taking place, appears to be the escape route. But is it?  \vspace{-0.15cm}

{\bf Three impasses} need to be clarified here. First: is there a (Bohmian) way to meaningfully talk about "unmeasured system" features, and even still be in accordance with phenomenology? Second: are these "unmeasured system" features experimentally accessible? And third: can these features be employed to operationally compute practical information, or are they mere "philosophical reliefs"?\vspace{-0.2cm}

\subsubsection*{3.1. Breaking Impasse 1: Speakable information of the "unmeasured" system}\vspace{-0.1cm}

Let us first clarify whether the information we obtain by measuring a quantum system is about the pre-measurement/"unmeasured" system or the post-measurement system. Consider an observable operator $\hat{B}=\sum_b b\ket{b}\bra{b}$, with $\{\ket{b}\}_b$ an orthonormal basis and $\ket{\psi}$ the wavefunction of the pre-measurement system. We have seen that (strongly) measuring $B$ will lead the system to the {\bf post}-measurement state $\ket{b}$ linked with the measured $b$, which will happen with a probability $|\bra{b}\ket{\psi}|^2$ due to the {\bf pre}-measurement state. Then, a single measurement tells us barely nothing about the {\bf pre-}measurement system. But if a "measurement", as Bell pointed out \cite{Bell}, has the connotation of unveiling or revealing information about the ({\bf pre-}measurement) system, it seems that it would be more proper to name this process an "experiment" rather than a "measurement". We can try to recover the name "measurement" with an ensemble of these "experiments" over identically prepared pre-measurement states $\ket{\psi}$. With them, we could estimate the (squared) magnitudes of the {\bf pre}-measurement projection-coefficients to each eigenstate $|\bra{b}\ket{\psi}|^2$ (e.g. using relative frequencies). Then, one could compute the expectation $\bra{\psi} \hat B \ket{\psi}=\sum_b b |\bra{b}\ket{\psi}|^2$, which is also a number dependent on the pre-measurement state $\ket{\psi}$. However, from a Copenhagen point of view, this number (say, the average energy or position of an electron) can only be interpreted as a property of the {\bf post}-measurement system, because by the eigenstate-eigenvalue link, only the post-measurement system can be attributed the observable $b$. When it comes to Bohmian mechanics, if $\hat{B}$ commutes with position $\hat{x}$, because the position $x$ is "speakable" at all times, the number $\bra{\psi} \hat B \ket{\psi}$ is the average property $B$ of the {\bf pre}-measurement system (as the simplest example, if $\hat{B}=\hat{x}$, it is the average Bohmian position of the unmeasured system). Yet, if $\hat{B}$ does not commute with $\hat{x}$ (say, if it is the momentum operator or a general Hamiltonian operator), it is unclear if the expectation $\bra{\psi} \hat B \ket{\psi}$ computed with the measured $b$, is a property of the pre-measurement system. In trying to clarify this, by linking the observable $B$ to the position $x$ of the Bohmian trajectories, which are "speakable", we will find the solution to the first impasse.

Given an arbitrary (Hermitian) operator $\hat{B}$, describing the observable $B$ for the subsystem S, with normalized EWF $\ket{\psi(t)}$, let us first blindly define the position function $C^{\psi}(\vec{x},t):={\bra{\vec{x}}\hat{B}\ket{\psi(t)}}/{\bra{\vec{x}}\ket{\psi(t)}}$. If we write the expected value for $\hat{B}$ as a function of $C^\psi(\vec{x},t)$, we get that\vspace{-0.15cm}
\begin{equation}\label{C}
\langle \hat{B}\rangle(t)= \bra{\psi(t)}\hat{B}\ket{\psi(t)}=\int \bra{\psi(t)} \ket{\vec{x}}\bra{\vec{x}}\hat{B}\ket{\psi(t)}dx =  \int |\psi(\vec{x},t)|^2C^\psi(\vec{x},t)dx.\vspace{-0.1cm}
\end{equation}
This means that the spatial average of the (possibly complex) $C^\psi(\vec{x},t)$ yields, at all times, the same expected value for the observable $B$ as that given by the Copenhagen theory. Now, let us define a real function $\B^\psi(\vec{x},t):=\mathbb{R}e\{C^{\psi}(\vec{x},t)\}$. Because $\hat{B}$ is an observable, its expected value will be a real number, such that $\langle \hat{B}\rangle=\mathbb{R}e\{\langle \hat{B}\rangle\}$. Thus, taking the real part of equation \eqref{C}, we get that\vspace{-0.1cm}
\begin{equation}
\langle \hat{B}\rangle(t)=\int |\psi(\vec{x},t)|^2\B^\psi(\vec{x},t)dx.\vspace{-0.1cm}
\end{equation}
We can link this with the set of Bohmian trajectories $\{\vec{x}^{\:\xi}(t)\}_{\xi\in\Sigma}$ sampled in independent repetitions of the experiment, by
using the Quantum Equilibrium hypothesis \cite{Absolute}, to get that $\langle \hat{B}\rangle(t)=\lim_{|\Sigma|\rightarrow \infty}\frac{1}{|\Sigma|} \sum_{\xi\in\Sigma} \B^\psi(\vec{x}^{\:\xi}(t),t)$. This means that the real number $\B^\psi(\vec{x}^{\;\xi}(t),t)$, related to the $\vec{\xi}$-th Bohmian trajectory of the "unmeasured" system, when averaged over the ensemble of possible trajectories, gives the same value as the operator's expectation value. That is, irrespective of whether or not we give the observable $B$ an ontological status, we can understand $\B^\psi(\vec{x},t)$ as a mathematical feature related to $B$ and linked to the Bohmian trajectory passing from $\vec{x}$ at time $t$ (in the "unmeasured" system). We will call $\B^\psi$, the "information $\B^\psi$" or the "information linked to $B$ and the Bohmian trajectory at $(\vec{x},t)$", to specify that we still mean nothing about its ontological or operational status.

For now, $\B^\psi$ appears to be just an ad-hoc function of the trajectories for the operator expected value to be satisfied. Let us see though, that it can be more than this. What would this number be for each trajectory if the system state, $\ket{\psi}$, was an eigenstate of $\hat{B}$ of eigenvalue $b$?\vspace{-0.1cm}
\begin{equation}
\B^\psi(\vec{x})=\mathbb{R}e\qty{ \frac{\bra{\vec{x}}\hat{B}\ket{\psi}}{\bra{\vec{x}}\ket{\psi}} } = \mathbb{R}e\qty{ \frac{\bra{\vec{x}}\ket{\psi}b}{\bra{\vec{x}}\ket{\psi}} }=b\quad \forall \vec{x}\vspace{-0.1cm}
\end{equation}
This suggests $\ket{\psi}$ is an eigenstate of $\hat{B}$ if and only if it is a state for which every Bohmian trajectory has the same value of the information $\B^\psi$. On the one hand, this tells us that the $b$ indicated by the dial of a projective measurement, can always be considered to be information linked to the Bohmian trajectory, even when its operator does not commute with position. On the other hand, in practice, it can be a tool to construct the operator $\hat{B}$ itself. Just define $\hat{B}$ in terms of $\B^\psi$, as the collection of states $\ket{b}$ in which all Bohmian trajectories have the same value $b$ for the information $\B^\psi$.

If the explicit shape of $\B^\psi$ had nothing to do with Bohmian mechanics, this reverse definition of $\hat{B}$ would be a circular definition. However, it turns out that if we set $\hat{B}$ to be the momentum operator $\hat{p_k}$ of the $k$-th degree of freedom, the trajectory information $\B^\psi(\vec{x},t)$ is exactly equal to the Bohmian momentum of the trajectory crossing $\vec{x}$: $m_k v_k(\vec{x},t)$ \cite{DevInPosition1}. If we set $\hat{B}$ to be the Hamiltonian operator $\hat{H}$, the information $\B^\psi(\vec{x},t)$ turns out to be exactly equal to the Bohmian energy (kinetic plus classical and quantum potentials \cite{JordiXavier}) of the trajectory crossing $\vec{x}$. One can see that the list of these "fortunate" matches for position functions that appeared to be designed only to satisfy the expectation values, goes on and on. This suggests that we can employ Bohmian mechanics to derive the expression for $\B^\psi$, thanks to its similarity with classical mechanics, and then define the related operator in those terms. Whether the information $\B^\psi$ has an ontological status or not, whether it is operational or not, this is already (numerically) useful, because there are observables, like the total (particle plus  displacement) current in a nano-device (plotted in Figure \ref{fig:fig}.b), for which there is no clear operator, but there is a clear Bohmian observable associated with it, as will be explained in detail later \cite{Pel, equiv}.

In a nutshell, since we placed no restriction on $\hat{B}$, we are mathematically safe to assume that at all times, each Bohmian trajectory $\vec{\xi}$, has a simultaneously determined value $\B^\psi(\vec{x}^{\:\xi}(t),t)$ linked to {\bf every} observable operator $\hat{B}$. Whether the information $\B^\psi(\vec{x}^{\:\xi}(t),t)$ reflects an \textbf{ontic property} (a property that the theory postulates to be part of the ontology) or not, is given by the quantum theory at hand. For example, we found that when $\B^\psi$ is linked to the momentum operator ${p}_k$, it is equal to the Bohmian momentum, which is an ontic property in Bohmian mechanics, but not in the Copenhagen theory. The key is that when $\B^\psi$ is equal to an ontic property, since the Bohmian trajectory exists in the absence of observation, $B$ becomes "speakable" with a well-defined value at all times. Importantly though, we saw that the information $\B^\psi$ is an equally well-defined number linked to each Bohmian trajectory, independently of the ontic character of $B$\footnote{The information $\B^\psi$ will actually evolve continuously as long as the wavefunction evolves unitarily (which in Bohmian mechanics does, as we saw). Then, if the wavefunction evolves from an eigenstate $\ket{b_1}$ to another $\ket{b_2}$ with eigenvalues $b_1\neq b_2$, $\B^\psi$ will take all the intermediate values not necessarily among the eigenvalues of $\hat{B}$. Thus, this suggests an interpretation in which the "quantization" of quantum mechanics is an apparent property, due to the fact that for a "proper" measurement, we require that a dial saying $b$ is compatible with a wavefunction $\ket{b}$ giving a von Neumann measurement $b$ with probability 1. That is, a wavefunction which has all its Bohmian trajectories with value $b$ for $\B^\psi$. Then, we would call it "quantum" because this delicate orchestration can only happen for a certain "quantized number" of wavefunctions (the eigenstates).}. Then, the fact (we will show now) that the $\B^\psi$ can be operationally obtained in a laboratory irrespective of their ontic character, will make the information $\B^\psi$ practically useful across the "unspeakables" and independently of the theory followed.
\vspace{-0.55cm}

 
\subsubsection*{3.2. Breaking Impasse 2: Is this "unmeasured" system information operational?}\vspace{-0.15cm}

If we could only obtain the information $\B^\psi$ in a laboratory when we used a strong von Neumann interaction, to force it to be an eigenvalue of $\hat{B}$, all this would limit us in practice in the same manner as the eigenstate-eigenvale link, even if we could now speak about these numbers in the absence of measurement. If so, we could not strictly say that the information $\B^\psi$ is an {\bf operational property}\footnote{A number that can be obtained in a laboratory with a well-defined protocol.} of the unmeasured system. However, it turns out we can actually obtain the "unmeasured" $\B^\psi$ even for a non-eigenstate pre-measurement system. The "how", explains the "cumbersome" definition $\B^\psi(\vec{x},t)=\mathbb{R}e\{\bra{\vec{x}}\hat{B}\ket{\psi}/\bra{\vec{x}}\ket{\psi}\}$. It turns out to be the protocol that naive classical experimentalists \cite{WisemanVel} would follow if they thought the system had a defined position, initially uncertain to them, and the only quantum knowledge they had was that measurement interactions spoil the system's natural subsequent evolution. In order to know the property $B$ of such a subsystem S (say, an electron) when it crosses $\vec{x}$, they would first couple an ancilla A to the subsystem S of EWF $\ket{\psi}$, through the measurement Hamiltonian $\bar{\mu}(t)\,\hat{p}_A\otimes\hat{B}$ but let the interaction strength $\mu$ be very small, such that the system state is only slightly perturbed. They would strongly measure the slightly entangled ancilla's position $z_B$ with a probability $P(z_B)$, getting a weak measurement about the property $B$ of S. Before the slightly perturbed system S further evolved, they would strongly measure its position $z_x$, with a certain conditional probability $P(z_x|z_B)$. Finally, they would average the weak measurements of $B$ for which the system S (the electron) was found at $\vec{x}$, in order to erase the noise introduced by the weakness of the coupling with A. If the averaged ensemble is large enough, the resulting conditional expectation will be equal (in the limit) to $\int z_B P(z_B|z_x)dz_B$, which as proven in \cite{DevInPosition2}, turns out to be roughly equal to $\B^\psi(\vec{x},t)$ (under some feasible experimental conditions). This is called a {\em position post-selected weak value} \cite{Weak}.

A naive experimentalist would not be surprised at all by such a "coincidence". One can consider all this was juggling with results of several observations. But, when the information $\B^\psi$ is an ontic property of the theory, one can legitimately say (under that theory), that the average weak measurements of $B$, for experiments in which the system (the electron) was at $\vec{x}$, gave $\B^\psi(\vec{x},t)$, because whenever the Bohmian trajectory (the electron) was at $\vec{x}$, it had indeed the property $\B^\psi(\vec{x},t)$. However, because we can follow this protocol in a lab for most observables $\hat{B}$, irrespective of their ontic state, $\B^\psi(\vec{x},t)$ is almost always\footnote{There is a (quite important) exception to keep in mind. Identical particles are always ontologically distinguishable by their trajectories in Bohmian mechanics. In the laboratory however, there is no technological means to tag each individual particle when dealing with many-body wave functions with exchange symmetry. For example, if we measure the Bohmian momentum along the physical $x_1$ direction for particles in a system of $M$ identical electrons, following this protocol, we will instead get the average: $\sum_{k=1}^M \frac{1}{M}\B^\psi_{(k)}(\vec{x}_1,...,\vec{x}_M)$ where $\B^\psi_{(k)}:=\mathbb{R}e\{\bra{\vec{x}_1,...,\vec{x}_M}\hat{Id}_{(1)}\cdots\hat{Id}_{(k-1)}\hat{p}_{1(k)}\hat{Id}_{(k+1)}\cdots\hat{Id}_{(M)}\ket{\psi}/\bra{\vec{x}_1,...,\vec{x}_M}\ket{\psi}\}$, with $\ket{\psi}$ the multi-electron wavefunction, following the notation in 2.1. Thus, the average $B$ for a multi-particle Bohmian trajectory is indeed operational (say, the sum of the total current contributions of the active region electrons, as discussed in the next paragraph), but the individual indistinguishable particle $\B^\Psi_{(k)}$ (like the individual electron current contributions) are not, even if they might be ontic properties within Bohmian mechanics.} an operational property \cite{DevInPosition1, DevInPosition2}.



\subsubsection*{3.3. Breaking Impasse 3: Is this information useful for a non-Bohmian?}
\vspace{-0.15cm}

Regardless of the followed quantum theory and whether one is ready to accept an ontological status for a certain information $\B^\psi$, its relation with expected values and the definition of the observable operator $\hat{B}$ is mathematically true. This has an important practical application that is equally useful for a non-Bohmian. The information $\B^\psi$ can be used to numerically predict the expected value of observables with no explicit definition as formal operators. For this, one can express the observable $B$ in the language of Bohmian mechanics to derive the shape of $\B^\psi(x,t)$, and then get the expected value of the operator related to $B$, by computing the trajectory ensemble average of $\B^\psi$. For example, this is how we predict the expected total electrical current (including the displacement current) crossing the active region of a two-terminal nano-device operating at high frequencies (THz) in the BITLLES simulator \cite{equiv, Pel}. We can define the contribution to the total current through a surface $\sigma$, due to the Bohmian trajectory of a $k$-th electron $\vec{x}_k^{\:\xi}(t)$ of charge $e$, as $I_k^{(\xi)}(t)=\int_\sigma \vec{J}^{\:(\xi)}(\vec{r},t)\cdot d\vec{s}+\int_\sigma \varepsilon(\vec{r},t)\pdv{\vec{E}^{\:(\xi)}(\vec{r},t)}{t}\cdot d\vec{s}$, where $\varepsilon(\vec{r},t)$ is the dielectric permittivity, $\vec{J}^{\:(\xi)}(\vec{r},t)=e\dv{\vec{x}_k^{\:\xi}(t)}{t}\delta(\vec{r}-\vec{x}_k^{\:\xi}(t))$ is the particle current density, and $\vec{E}^{\:(\xi)}(\vec{r},t)$ is the electric field generated by the electron, as a solution to Gauss' equation. The sum of these contributions, $I^{(\xi)}(t)=\sum_k I^{(\xi)}_k(t)$, will be the total Bohmian current at the surface $\sigma$ for the $\xi$-th experiment in the ensemble $\{\vec{x}^{\:\xi}(t)\}_{\xi\in \Sigma}$. The phenomenological expectation\footnote{In addition, the fluctuations of the current $I^{(\xi)}(t)$ in each experiment (due to partition and thermal noises) provide meaningful time-correlations, even in non-Markovian scenarios (as the one depicted in Figure \ref{fig:fig}).} of a total current operator $\hat{I}$ will then be computable as the ensemble average of these currents, since $lim_{|\Sigma|\rightarrow \infty}\frac{1}{|\Sigma|} \sum_{\xi\in\Sigma} I^{(\xi)}(t)=\langle \hat{I}\rangle(t)$, by the Quantum Equilibrium hypothesis.

On the other hand, it is also true that the information $\B^\psi$ is a number that, no matter the followed interpretation of quantum mechanics, characterizes or describes the theoretical pre-measurement wavefunction $\ket{\psi}$, and it is true it is typically experimentally obtainable. This readily suggests that it can be pragmatically employed to characterize a closed quantum system, just like a tomography of the probability density is useful to characterize the closed system wavefunction at a certain time. Following this, we find an interesting application for those $\B^\psi$ that happen to be operational, in the puzzling search of non-contextuality for closed-system metrics involving two different times \cite{DevInPosition1}.

For example, the problem of non-contextual two-time correlations of dynamical variables for a wavefunction $\ket{\psi(t)}$, can be circumvented as follows. Given a big enough set of trajectories $\{\vec{x}^{\:\xi}(t)\}_{\xi\in \Sigma}$ associated with the functions $\B^\psi(\vec{x},t)$ and $\mathcal{F}^\psi(\vec{x},t)$, following Quantum Equilibrium \cite{Absolute}, we could define their correlation as\vspace{-0.1cm}
\begin{equation}
\langle B(t_2)F(t_1)\rangle := \lim_{|\Sigma|\rightarrow \infty}\frac{1}{|\Sigma|} \sum_{\xi\in\Sigma} \B^\psi(\vec{x}^{\:\xi}(t_2),t_2)\mathcal{F}^\psi(\vec{x}^{\:\xi}(t_1),t_1) =\vspace{-0.05cm}
\end{equation}
$$
=  \int |\psi(\vec{\xi},0)|^2\ \mathbb{R}e\qty[ \frac{\bra{\vec{x}^{\:\xi}(t_2)}\hat{B}\ket{\psi(t_2)}}{\bra{\vec{x}^{\:\xi}(t_2)}\ket{\psi(t_2)}} ] \mathbb{R}e\qty[ \frac{\bra{\vec{x}^{\:\xi}(t_1)}\hat{F}\ket{\psi(t_1)}}{\bra{\vec{x}^{\:\xi}(t_1)}\ket{\psi(t_1)}} ]d\xi.
$$

 In a similar way, we can solve the problems concerning a quantum work definition, just as done by \cite{work1, work2}. First note that given a general system Hamiltonian $\hat{H}=\sum_k \frac{-\hbar^2}{2m_k}\pdv[2]{}{x_k}+V(\vec{x},t)$, we get
\begin{equation}
\mathcal{H}^\psi(\vec{x}^{\:\xi}(t),t) := \mathbb{R}e\qty[ \frac{\bra{\vec{x}^{\:\xi}(t)}\hat{H}\ket{\psi(t)}}{\bra{\vec{x}^{\:\xi}(t)}\ket{\psi(t)}} ] = \sum_{k=1}^n\frac{1}{2}m_kv_k(\vec{x}^{\:\xi}(t),t)^2+V(\vec{x}^{\:\xi}(t),t)+Q(\vec{x}^{\:\xi}(t),t),
\end{equation}
with $Q$ the well-known Bohmian quantum potential \cite{Holland, Durr, JordiXavier}. This proves $\mathcal{H}^\psi(\vec{x}^{\:\xi}(t),t)$ is, as anticipated, the total Bohmian energy of the $\vec{\xi}$-th trajectory at time $t$. Then, following classical mechanics, we can compute its associated Bohmian work as the energy difference (if conservative, or else employing an integral): $\mathcal{W}^{(\xi)}(t_1,t_2)= \mathcal{H}^\psi(\vec{x}^{\:\xi}(t_2),t_2)-\mathcal{H}^\psi(\vec{x}^{\:\xi}(t_1),t_1)$. As a result, a well-defined non-contextual definition of the quantum work could be the ensemble average of the trajectory works,
\begin{equation}
\langle W(t_1,t_2)\rangle = \lim_{|\Sigma|\rightarrow \infty}\frac{1}{|\Sigma|} \sum_{\xi\in\Sigma} \qty(\mathcal{H}^\psi(\vec{x}^{\:\xi}(t_2),t_2)-\mathcal{H}^\psi(\vec{x}^{\:\xi}(t_1),t_1)).
\end{equation}
Finally, we could give a reasonable Bohmian answer to the pathological search of an "unmeasured" dwell time, as the expected time spent by the Bohmian trajectory of the electron within the active region $\Gamma\subset \R^3$. Mathematically, the dwell time $\tau$ for the $\vec{\xi}$-th trajectory of the $k$-th electron with EWF $\psi^\xi(\vec{x}_k,t)$ is by definition given by the Lebesgue integral: $\tau^{( \xi)}= \int_{0}^\infty  dt \int_\Gamma \delta(\vec{r}-\vec{x}_k^{\:\xi}(t)) dr$. This makes the expected time $\langle \tau\rangle$ to be given by the Quantum Equilibrium hypothesis as an integral that is already employed to predict the dwell time,\vspace{-0.05cm}
\begin{equation}
\langle \tau \rangle = \lim_{|\Sigma|\rightarrow \infty}\frac{1}{|\Sigma|} \sum_{\xi\in\Sigma} \tau^{(\xi)} = \int_{0}^\infty dt \int_\Gamma |\psi^\xi(\vec{r},t)|^2dr.\vspace{-0.14cm}
\end{equation}

To conclude the section and link it with the discussion on non-Markovian SSEs, notice that, because in the non-Markovian case, the trajectory for the "unraveled" environment observable (what we denoted by $w(t)$), can no longer be interpreted as the result of a continuous measurement of the environment, it represents an unmeasured observable of the environment. Thus, this is readily a, perhaps unintended, application of $\B^\psi$-like properties, which happen to be central to simulate the most general quantum systems that openly interact with many environmental degrees of freedom. \vspace{-0.2cm}

\subsection*{4. Conclusions}\vspace{-0.15cm}

In this chapter, we have seen that inherently Bohmian concepts like the CWF or position post-selected weak values are indeed usable pragmatically as practical tools in the computation of phenomenologically accessible elements, such as the reduced density matrix, expectation values or time correlations. Therefore, with this chapter, we refute the main criticism to the Bohmian theory, by which the trajectories are "unnecesary embellishments" of the orthodox theory, with no practical use.

In fact, we have seen with several examples that dressing these practical tools with the Bohmian interpretation makes them even more practical, by aiding in the search of SSEs or observable operators. But then, if we can use Bohmian concepts as a tool, why not include them in our vocabulary? Not only for their problem-solving utility, but also because they can provide us ontological relief in front of the purely phenomenological Copenhagen view. This renewed appeal of the Bohmian theory is clearly motivated by a time when no engineer is really capable of accepting the "unspeakable" quantum reality \cite{where, consp}. It must be noted thought, that not even great parents of the quantum theory were ready to restrict themselves to the Copenhagen doctrine. For example, regarding the first section, von Neumann in his seminal book \cite{vonNeumann} explains that the collapse law is to be understood as an effective process that should be possible to be considered at an arbitrary point between the subsystem and the macroscopic device, instead of considering it to be a physical phenomenon \cite{NeumannNoCollapse}. Linking this with the discussion on the contextuality of the quantum measurement that avoids a simple elucidation of the pre-measurement system, Bohr himself did not believe in a physical collapse and instead assigned it to the contextuality of experimental protocols \cite{Dirac} in terms of macroscopic devices \cite{Bohr}. The claims of both scientists are naturally satisfied by the Bohmian description of the measurement, as we showed in this review, which does not need to postulate any collapse law. When it comes to the second section, it was H. M. Wiseman who pointed out that SSEs for non-Markovian systems implied the usage of CWFs of modal theories \cite{interpretSSE, NMisModal} and who suggested the first formal position SSEs for such open quantum systems \cite{WisemanSSE}. Finally, regarding the discussion on the unspeakables of the third section, Dirac himself was an exemplary physicist that employed "unspeakable unmeasured" system properties in the formulation of his major contributions to physics, leaving questioned the "observability doctrine" of the Copenhagen interpretation \cite{Dirac}.

With all this, we might be wondering when will physicists decide to break the limiting walls around (non-relativistic) quantum mechanics, which are voluntarily taught to new generations of scientists every day. There is a coherent and pedagogical narrative (the Bohmian one) to explain it all while avoiding paradoxes and disjunctives with classical intuitions, just working with real and causal microscopic properties. This is a narrative that actually proves to be practically useful by offering additional tools to the Copenhagen theory. Shouldn't we someday include it in the standard program of quantum mechanics taught in our Universities? Only time will tell.


{\small
\bibliographystyle{ieeetr}

\bibliography{Bibliography}
}




\end{document}






